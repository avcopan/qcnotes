\documentclass[11pt,fleqn]{article}
\usepackage[cm]{fullpage}
\usepackage{mathtools} %includes amsmath
\usepackage{amsfonts}
\usepackage{bm}
\usepackage{xfrac}
\usepackage{url}
%greek letters
\renewcommand{\a}{\alpha}    %alpha
\renewcommand{\b}{\beta}     %beta
\newcommand{\g}{\gamma}      %gamma
\newcommand{\G}{\Gamma}      %Gamma
\renewcommand{\d}{\delta}    %delta
\newcommand{\D}{\Delta}      %Delta
\newcommand{\e}{\varepsilon} %epsilon
\newcommand{\ev}{\epsilon}   %epsilon*
\newcommand{\z}{\zeta}       %zeta
\newcommand{\h}{\eta}        %eta
\renewcommand{\th}{\theta}   %theta
\newcommand{\Th}{\Theta}     %Theta
\newcommand{\io}{\iota}      %iota
\renewcommand{\k}{\kappa}    %kappa
\newcommand{\la}{\lambda}    %lambda
\newcommand{\La}{\Lambda}    %Lambda
\newcommand{\m}{\mu}         %mu
\newcommand{\n}{\nu}         %nu %xi %Xi %pi %Pi
\newcommand{\p}{\rho}        %rho
\newcommand{\si}{\sigma}     %sigma
\newcommand{\siv}{\varsigma} %sigma*
\newcommand{\Si}{\Sigma}     %Sigma
\renewcommand{\t}{\tau}      %tau
\newcommand{\up}{\upsilon}   %upsilon
\newcommand{\f}{\phi}        %phi
\newcommand{\F}{\Phi}        %Phi
\newcommand{\x}{\chi}        %chi
\newcommand{\y}{\psi}        %psi
\newcommand{\Y}{\Psi}        %Psi
\newcommand{\w}{\omega}      %omega
\newcommand{\W}{\Omega}      %Omega
%ornaments
\newcommand{\eth}{\ensuremath{^\text{th}}}
\newcommand{\rst}{\ensuremath{^\text{st}}}
\newcommand{\ond}{\ensuremath{^\text{nd}}}
\newcommand{\ord}[1]{\ensuremath{^{(#1)}}}
\newcommand{\dg}{\ensuremath{^\dagger}}
\newcommand{\bigo}{\ensuremath{\mathcal{O}}}
\newcommand{\tl}{\ensuremath{\tilde}}
\newcommand{\ol}[1]{\ensuremath{\overline{#1}}}
\newcommand{\ul}[1]{\ensuremath{\underline{#1}}}
\newcommand{\op}[1]{\ensuremath{\hat{#1}}}
\newcommand{\ot}{\ensuremath{\otimes}}
\newcommand{\wg}{\ensuremath{\wedge}}
%text
\newcommand{\tr}{\ensuremath{\hspace{1pt}\mathrm{tr}\hspace{1pt}}}
\newcommand{\Alt}{\ensuremath{\mathrm{Alt}}}
\newcommand{\sgn}{\ensuremath{\mathrm{sgn}}}
\newcommand{\occ}{\ensuremath{\mathrm{occ}}}
\newcommand{\vir}{\ensuremath{\mathrm{vir}}}
\newcommand{\spn}{\ensuremath{\mathrm{span}}}
\newcommand{\vac}{\ensuremath{\mathrm{vac}}}
\newcommand{\bs}{\ensuremath{\text{\textbackslash}}}
\newcommand{\im}{\ensuremath{\mathrm{im}\hspace{1pt}}}
\renewcommand{\sp}{\hspace{30pt}}
%dots
\newcommand{\ld}{\ensuremath{\ldots}}
\newcommand{\cd}{\ensuremath{\cdots}}
\newcommand{\vd}{\ensuremath{\vdots}}
\newcommand{\dd}{\ensuremath{\ddots}}
\newcommand{\etc}{\ensuremath{\mathinner{\mkern-1mu\cdotp\mkern-2mu\cdotp\mkern-2mu\cdotp\mkern-1mu}}}
%fonts
\newcommand{\bmit}[1]{{\bfseries\itshape\mathversion{bold}#1}}
\newcommand{\mc}[1]{\ensuremath{\mathcal{#1}}}
\newcommand{\mb}[1]{\ensuremath{\mathbb{#1}}}
\newcommand{\mf}[1]{\ensuremath{\mathfrak{#1}}}
\newcommand{\mr}[1]{\ensuremath{\mathrm{#1}}}
\newcommand{\bo}[1]{\ensuremath{\mathbf{#1}}}
%styles
\newcommand{\ts}{\textstyle}
\newcommand{\ds}{\displaystyle}
\newcommand{\phsub}{\ensuremath{_{\phantom{p}}}}
\newcommand{\phsup}{\ensuremath{^{\phantom{p}}}}
%fractions, derivatives, parentheses, brackets, etc.
\newcommand{\pr}[1]{\ensuremath{\left(#1\right)}}
\newcommand{\brk}[1]{\ensuremath{\left[#1\right]}}
\newcommand{\fr}[2]{\ensuremath{\dfrac{#1}{#2}}}
\newcommand{\pd}[2]{\ensuremath{\frac{\partial#1}{\partial#2}}}
\newcommand{\pt}{\ensuremath{\partial}}
\newcommand{\br}[1]{\ensuremath{\langle#1|}}
\newcommand{\kt}[1]{\ensuremath{|#1\rangle}}
\newcommand{\ip}[1]{\ensuremath{\langle#1\rangle}}
\newcommand{\NO}[1]{\ensuremath{{\bm{:}}#1{}{\bm{:}}}}
\newcommand{\floor}[1]{\ensuremath{\left\lfloor#1\right\rfloor}}
\newcommand{\ceil}[1]{\ensuremath{\left\lceil#1\right\rceil}}
\usepackage{stackengine}
\newcommand{\GNO}[1]{\setstackgap{S}{0.7pt}\ensuremath{\Shortstack{\textbf{.} \textbf{.} \textbf{.}}#1\Shortstack{\textbf{.} \textbf{.} \textbf{.}}}}
\newcommand{\cmtr}[2]{\ensuremath{[\cdot,#2]^{#1}}}
\newcommand{\cmtl}[2]{\ensuremath{[#2,\cdot]^{#1}}}
%structures
\newcommand{\eqn}[1]{(\ref{#1})}
\newcommand{\ma}[1]{\ensuremath{\begin{bmatrix}#1\end{bmatrix}}}
\newcommand{\ar}[1]{\ensuremath{\begin{matrix}#1\end{matrix}}}
\newcommand{\miniar}[1]{\ensuremath{\begin{smallmatrix}#1\end{smallmatrix}}}
%contractions
\usepackage{simplewick}
\usepackage[nomessages]{fp}
\newcommand{\ctr}[6][0]{\FPeval\height{0.6+#1*0.5}\ensuremath{\contraction[\height ex]{#2}{#3}{#4}{#5}}}
\newcommand{\ccr}[4]{\ctr[0.7]{#1}{#2}{#3}{#4}{}\ctr{#1}{#2}{#3}{#4}}
\usepackage{calc}
\makeatletter
\def\@hspace#1{\begingroup\setlength\dimen@{#1}\hskip\dimen@\endgroup}
\makeatother
\newcommand{\halfphantom}[1]{\hspace{\widthof{#1}*\real{0.5}}}
\newcommand{\fullctr}[1]{\ensuremath{\contraction[0.5ex]{}{\vphantom{#1}}{\hphantom{#1}}{}{}{}\contraction[0.5ex]{}{\vphantom{#1}}{\halfphantom{#1}}{}{}{}#1}}
%math sections
\usepackage{cleveref}
\usepackage{amsthm}
\usepackage{thmtools}
\declaretheoremstyle[spaceabove=10pt,spacebelow=10pt,bodyfont=\small]{mystyle}
\theoremstyle{mystyle}
\newtheorem{dfn}{Definition}[section]
\crefname{dfn}{definition}{definitions}
\Crefname{dfn}{Def}{Defs}
\newtheorem{thm}{Theorem}[section]
\crefname{thm}{theorem}{theorems}
\Crefname{thm}{Thm}{Thms}
\newtheorem{cor}{Corollary}[section]
\crefname{cor}{corollary}{corollaries}
\Crefname{cor}{Cor}{Cors}
\newtheorem{lem}{Lemma}[section]
\crefname{lem}{lemma}{lemmas}
\Crefname{lem}{Lem}{Lems}
\newtheorem{rmk}{Remark}[section]
\crefname{rmk}{remark}{remarks}
\Crefname{rmk}{Rmk}{Rmks}
\newtheorem{pro}{Proposition}[section]
\crefname{pro}{proposition}{propositions}
\Crefname{pro}{Prop}{Props}
\newtheorem{ntt}{Notation}[section]
\crefname{ntt}{notation}{notations}
\Crefname{ntt}{Notation}{Notations}

\usepackage{cancel}
\usepackage{scrextend}

\newcommand{\hole}{\circ}
\newcommand{\ptcl}{\bullet}
\newcommand{\circled}[1]{\raisebox{.5pt}{\textcircled{\raisebox{-.9pt}{#1}}}}

\usepackage{tikz}
\usetikzlibrary{positioning}
\usetikzlibrary{backgrounds}
\usetikzlibrary{shapes}
\usetikzlibrary{decorations.markings}
%\usetikzlibrary{positioning}
\newcommand*{\halfway}[1]{#1*\pgfdecoratedpathlength+1pt}
\tikzset{
->-/.style={thick,decoration={
  markings,
  mark=at position \halfway{#1} with {\arrow{>}}},
  postaction={decorate}},
-<-/.style={thick,decoration={
  markings,
  mark=at position \halfway{#1} with {\arrow{<}}},
  postaction={decorate}},
}
\newcommand{\background}[1]{
  \begin{scope}[on background layer]
    #1
  \end{scope}
}
\newcommand{\padborder}[1]{
  \draw[line width=#1,opacity=0] (current bounding box.south east) rectangle (current bounding box.north west);
}
\newcommand{\tikpic}[2][5pt]{
  \begin{tikzpicture}[baseline=-2.5pt]
  #2
  \padborder{#1}
  \end{tikzpicture}
}
\newcommand{\interactionlabel}[3]{\path #3 node (#1) [draw,regular polygon, regular polygon sides=4,inner sep=0pt] {\small{#2}}}
\newcommand{\interactionpoint}[3][black]{\path #3 node (#2) [draw,shape=circle,fill=#1,inner sep=0pt,minimum size=0.5ex] {}}
\newcommand{\labeledinteraction}[6][black]{
  \interactionlabel{#3}{#4}{#5};
  \draw[dashed] (#3) to #6 to ++(#2-1,0);
  \foreach \i in {1,...,#2}{
    \interactionpoint[#1]{#3\i}{#6+(\i-1,0)};
  }
}
\newcommand{\oneelinteraction}[5][black]{
  \path #4 node (#2) [draw,circle,inner sep=0pt] {\small{#3}};
  \draw[dashed] (#2) to #5;
  \unlabeledinteraction[#1]{1}{#2}{#5}
}
\newcommand{\unlabeledinteraction}[4][black]{
  \draw[dashed] #4 to ++(#2-1,0);
  \foreach \i in {1,...,#2}{
    \interactionpoint[#1]{#3\i}{#4+(\i-1,0)};
  }
}
\newcommand{\amplitudeinteraction}[4][black]{
  \draw[shorten >=-5pt,shorten <=-5pt] #4 to ++(#2-1,0);
  \foreach \i in {1,...,#2}{
    \interactionpoint[#1]{#3\i}{#4+(\i-1,0)};
  }
}

%%%DOCUMENT%%%
\begin{document}


\section*{CID equations}

\subsection*{Derivation in KM tensor notation}

The electronic Hamiltonian can be expressed as follows.
\begin{align}
&&
  H_e
=
  E_{\text{HF}}
+
  H_c
&&
  E_{\text{HF}}
=
  \ip{\F|H_e|\F}
&&
  H_c
=
  f_p^q
  \tl{a}_q^p
+
  \tfrac{1}{4}
  \ol{g}_{pq}^{rs}
  \tl{a}_{rs}^{pq}
\end{align}
The CID ansatz parametrizes the wavefunction as $\Y\approx(c_0+\op{C}_2)\F$ where\footnote{We write the CI coefficients here as $c_{ab}^{ij}$ instead of $c_{ij}^{ab}$ in order to be consistent with Einstein notation.} $\op{C}_2=\tfrac{1}{4}c_{ab}^{ij}a_{ij}^{ab}$, and the CID correlation energy can obtained by solving the following system of linear equations
\begin{align}
\label{cid-linear-equations-1}
&&
  \ip{\F|H_c(c_0+\op{C}_2)|\F}
=&\
  E_cc_0
&&
  \implies
&
  \tfrac{1}{4}
  \ip{\F|H_c|\F_{kl}^{cd}}c_{cd}^{kl}
=&\
  E_cc_0
\\
\label{cid-linear-equations-2}
&&
  \ip{\F_{ij}^{ab}|H_c(c_0+\op{C}_2)|\F}
=&\
  E_cc_{ab}^{ij}
&&
  \implies
&
  \ip{\F_{ij}^{ab}|H_c|\F}c_0
+
  \tfrac{1}{4}
  \ip{\F_{ij}^{ab}|H_c|\F_{kl}^{cd}}c_{cd}^{kl}
=&\
  E_cc_{ab}^{ij}
\end{align}
which come from projecting the Schr\"odinger equation in the form $H_c\Y=E_c\Y$ by $\F$ and $\F_{ij}^{ab}$.
This is equivalent to solving for a root of the CID matrix
\begin{align}
&&
  \ma{
    \ip{\F|H_c|\F}&\ip{\F|H_c|\bo{D}}\\
    \ip{\bo{D}|H_c|\F}&\ip{\bo{D}|H_c|\bo{D}}
  }
  \ma{c_0\\\bo{c}_{\bo{D}}}
=
  E_c
  \ma{c_0\\\bo{c}_{\bo{D}}}
&&
  \bo{D}=\{\F_{ij}^{ab}\}, \ 
  \bo{c}_{\bo{D}}=\{c_{ij}^{ab}\}
\end{align}
where the lowest root corresponds to the ground state correlation energy.
The relevant matrix elements are
\begin{align}
\label{cid-matrix-elements}
&&
  \ip{\F|H_c|\F}
=
  0
&&
  \ip{\F|H_c|\F_{ij}^{ab}}
=
  \ol{g}_{ij}^{ab}
&&
  \ip{\F_{ij}^{ab}|H_c|\F_{kl}^{cd}}
=
  f_p^q
  (\tl{a}_{ab}^{ij}\tl{a}_q^p\tl{a}_{kl}^{cd})_{\text{f.c.}}
+
  \tfrac{1}{4}
  \ol{g}_{pq}^{rs}
  (\tl{a}_{ab}^{ij}\tl{a}_{rs}^{pq}\tl{a}_{kl}^{cd})_{\text{f.c.}}
\end{align}
where $(\tl{a}_{ab}^{ij}\tl{a}_q^p\tl{a}_{kl}^{cd})_{\text{f.c.}}$ and $(\tl{a}_{ab}^{ij}\tl{a}_{rs}^{pq}\tl{a}_{kl}^{cd})_{\text{f.c.}}$ can be determined using Wick's theorem
\begin{align*}
  (\tl{a}_{ab}^{ij}\tl{a}_q^p\tl{a}_{kl}^{cd})_{\text{f.c.}}
=&\
  \op{P}^{(c/d)}
        _{(a/b|k/l)}
  \GNO{
    \tl{a}_{a^{\hole1}b^{\hole3}}
          ^{i^{\ptcl1}j^{\ptcl2}}
    \tl{a}_{q^{\hole2}}
          ^{p^{\hole1}}
    \tl{a}_{k^{\ptcl1}l^{\ptcl2}}
          ^{c^{\hole2}d^{\hole3}}
  }
+
  \op{P}^{(i/j|c/d)}
        _{(k/l)}
  \GNO{
    \tl{a}_{a^{\hole1}b^{\hole2}}
          ^{i^{\ptcl1}j^{\ptcl3}}
    \tl{a}_{q^{\ptcl1}}
          ^{p^{\ptcl2}}
    \tl{a}_{k^{\ptcl2}l^{\ptcl3}}
          ^{c^{\hole1}d^{\hole2}}
  }
\\=&\
-
  \op{P}^{(c/d)}
        _{(a/b|k/l)}
  \tl{a}_{a^{\hole1}q^{\hole2}b^{\hole3}k^{\ptcl1}l^{\ptcl2}}
        ^{p^{\hole1}c^{\hole2}d^{\hole3}i^{\ptcl1}j^{\ptcl2}}
-
  \op{P}^{(i/j|c/d)}
        _{(k/l)}
  \tl{a}_{q^{\ptcl1}k^{\ptcl2}l^{\ptcl3}a^{\hole1}b^{\hole2}}
        ^{i^{\ptcl1}p^{\ptcl2}j^{\ptcl3}c^{\hole1}d^{\hole2}}
\\=&\
  \op{P}^{(c/d)}
        _{(a/b|k/l)}
  \h_a^p\h_q^c\h_b^d\k_k^i\k_l^j
-
  \op{P}^{(i/j|c/d)}
        _{(k/l)}
  \k_q^i\k_k^p\k_l^j\h_a^c\h_b^d
\\
  (\tl{a}_{ab}^{ij}\tl{a}_{rs}^{pq}\tl{a}_{kl}^{cd})_{\text{f.c.}}
=&\
  \op{P}^{(c/d)}
        _{(a/b|k/l)}
  \GNO{
    \tl{a}_{a^{\hole1}b^{\hole2}}
          ^{i^{\ptcl1}j^{\ptcl2}}
    \tl{a}_{r^{\hole3}s^{\hole4}}
          ^{p^{\hole1}q^{\hole2}}
    \tl{a}_{k^{\ptcl1}l^{\ptcl2}}
          ^{c^{\hole3}d^{\hole4}}
  }
+
  \op{P}^{(i/j|c/d)}
        _{(k/l)}
  \GNO{
    \tl{a}_{a^{\hole1}b^{\hole2}}
          ^{i^{\ptcl1}j^{\ptcl2}}
    \tl{a}_{r^{\ptcl1}s^{\ptcl2}}
          ^{p^{\ptcl3}q^{\ptcl4}}
    \tl{a}_{k^{\ptcl3}l^{\ptcl4}}
          ^{c^{\hole1}d^{\hole2}}
  }
\\&\
+
  \op{P}^{(p/q|i/j|c/d)}
        _{(r/s|k/l|a/b)}
  \GNO{
    \tl{a}_{a^{\hole1}b^{\hole3}}
          ^{i^{\ptcl1}j^{\ptcl3}}
    \tl{a}_{r^{\ptcl1}s^{\hole2}}
          ^{p^{\ptcl2}q^{\hole1}}
    \tl{a}_{k^{\ptcl2}l^{\ptcl3}}
          ^{c^{\hole2}d^{\hole3}}
  }
\\=&\
  \op{P}^{(c/d)}
        _{(a/b|k/l)}
  \tl{a}_{a^{\hole1}b^{\hole2}r^{\hole3}s^{\hole4}
          k^{\ptcl1}l^{\ptcl2}}
        ^{p^{\hole1}q^{\hole2}c^{\hole3}d^{\hole4}
          i^{\ptcl1}j^{\ptcl2}}
+
  \op{P}^{(i/j|c/d)}
        _{(k/l)}
  \tl{a}_{r^{\ptcl1}s^{\ptcl2}k^{\ptcl3}l^{\ptcl4}
          a^{\hole1}b^{\hole2}}
        ^{i^{\ptcl1}j^{\ptcl2}p^{\ptcl3}q^{\ptcl4}
          c^{\hole1}d^{\hole2}}
+
  \op{P}^{(p/q|i/j|c/d)}
        _{(r/s|k/l|a/b)}
  \tl{a}_{r^{\ptcl1}k^{\ptcl2}l^{\ptcl3}
          a^{\hole1}s^{\hole2}b^{\hole3}}
        ^{i^{\ptcl1}p^{\ptcl2}j^{\ptcl3}
          q^{\hole1}c^{\hole2}d^{\hole3}}
\\=&\
  \op{P}^{(c/d)}
        _{(a/b|k/l)}
  \h_a^p\h_b^q\h_r^c\h_s^d
  \k_k^i\k_l^j
+
  \op{P}^{(i/j|c/d)}
        _{(k/l)}
  \k_r^i\k_s^j\k_k^p\k_l^q
  \h_a^c\h_b^d
-
  \op{P}^{(p/q|i/j|c/d)}
        _{(r/s|k/l|a/b)}
  \k_r^i\k_k^p\k_l^j
  \h_a^q\h_s^c\h_b^d
\end{align*}
giving
\begin{align}
\label{doubles-block}
  \ip{\F_{ij}^{ab}|H_c|\F_{kl}^{cd}}
=
  \op{P}^{(c/d)}_{(a/b|k/l)}
  f_a^c\d_b^d\d_k^i\d_l^j
-
  \op{P}^{(i/j|c/d)}_{(k/l)}
  f_k^i\d_l^j\d_a^c\d_b^d
+
  \op{P}_{(k/l)}
  \ol{g}_{ab}^{cd}\d_k^i\d_l^j
+
  \op{P}^{(c/d)}
  \ol{g}_{kl}^{ij}\d_a^c\d_b^d
-
  \op{P}^{(i/j|c/d)}_{(k/l|a/b)}
  \ol{g}_{ka}^{ic}\d_l^j\d_b^d
\end{align}
for the matrix elements of the doubles block, $\ip{\bo{D}|H_c|\bo{D}}$.
Efficient CI implementations utilize a ``direct algorithm'', in which the Hamiltonian matrix $\bo{\tl{H}}=[\ip{\F_P|H_c|\F_Q}]$ is never constructed and stored in memory.
Instead, the product $\bo{\tl{H}}\bo{c}$ is computed directly (hence the name) for a given trial vector $\bo{c}$.
This matrix-vector product is called the ``sigma vector'', $\bm\si$, with elements $\si_P=\sum_Q\ip{\F_P|H_c|\F_Q}c_Q$.
Elements of the CID sigma vector are given by the left-hand sides of equations \ref{cid-linear-equations-1} and \ref{cid-linear-equations-2}.
Using equations \ref{cid-matrix-elements} and \ref{doubles-block}, the working expressions for $\si_0$ and $\si_{ab}^{ij}$ are as follows.
\begin{align}
  \si_0
=&\
  \tfrac{1}{4}
  \ol{g}_{kl}^{cd}c_{cd}^{kl}
&&
  \pr{\si_0\equiv\tfrac{1}{4}\ip{\F|H_c|\F_{kl}^{cd}}c_{cd}^{kl}}
\\
  \si_{ab}^{ij}
=&\
  \ol{g}_{ab}^{ij}c_0
+
  \op{P}_{(a/b)}
  f_a^c c_{cb}^{ij}
-
  \op{P}^{(i/j)}
  f_k^ic_{ab}^{kj}
+
  \tfrac{1}{2}
  \ol{g}_{ab}^{cd}c_{cd}^{ij}
+
  \tfrac{1}{2}
  \ol{g}_{kl}^{ij}
  c_{ab}^{kl}
-
  \op{P}^{(i/j)}_{(a/b)}
  \ol{g}_{ka}^{ic}c_{cb}^{kj}
&&
  \pr{\si_{ab}^{ij}\equiv\tfrac{1}{4}\ip{\F_{ij}^{ab}|H_c|\F_{kl}^{cd}}c_{cd}^{kl}}
\end{align}


\subsection*{Derivation in diagrammatic notation}

In diagrammatic notation, $H_c$ is given by the following.
\begin{align}
&&
  H_c
=&\
\tikpic{
  \oneelinteraction[white]{f}{$\times$}{(-0.75,0)}{(0,0)};
  \background{
    \draw[->-=0.5] (f1) to ++(0,+0.5);
    \draw[-<-=0.5] (f1) to ++(0,-0.5);
  }
}
+
\tikpic{
  \unlabeledinteraction[white]{2}{g}{(0,0)};
  \background{
    \draw[->-=0.5] (g1) to ++(0,+0.5);
    \draw[-<-=0.5] (g1) to ++(0,-0.5);
    \draw[->-=0.5] (g2) to ++(0,+0.5);
    \draw[-<-=0.5] (g2) to ++(0,-0.5);
  }
}
&
\tikpic{
  \oneelinteraction[white]{f}{$\times$}{(-0.75,0)}{(0,0)};
  \background{
    \draw[->-=0.5] (f1) to ++(0,+0.5);
    \draw[-<-=0.5] (f1) to ++(0,-0.5);
  }
}
=&\
\tikpic{
  \oneelinteraction{f}{$\times$}{(-0.75,0)}{(0,0)};
  \background{
    \draw[->-=0.5] (f1) to ++(0,+0.5);
    \draw[-<-=0.5] (f1) to ++(0,-0.5);
  }
}
+
\tikpic{
  \oneelinteraction{f}{$\times$}{(-0.75,0)}{(0,0)};
  \background{
    \draw[->-=0.5] (f1) to ++(-0.25,0.5);
    \draw[-<-=0.5] (f1) to ++(+0.25,0.5);
  }
}
+
\tikpic{
  \oneelinteraction{f}{$\times$}{(-0.75,0)}{(0,0)};
  \background{
    \draw[->-=0.5] (f1) to ++(-0.25,-0.5);
    \draw[-<-=0.5] (f1) to ++(+0.25,-0.5);
  }
}
+
\tikpic{
  \oneelinteraction{f}{$\times$}{(-0.75,0)}{(0,0)};
  \background{
    \draw[-<-=0.5] (f1) to ++(0,+0.5);
    \draw[->-=0.5] (f1) to ++(0,-0.5);
  }
}
\\
\nonumber
&&&&
\tikpic{
  \unlabeledinteraction[white]{2}{g}{(0,0)};
  \background{
    \draw[->-=0.5] (g1) to ++(0,+0.5);
    \draw[-<-=0.5] (g1) to ++(0,-0.5);
    \draw[->-=0.5] (g2) to ++(0,+0.5);
    \draw[-<-=0.5] (g2) to ++(0,-0.5);
  }
}
=&\
\tikpic{
  \unlabeledinteraction{2}{g}{(0,0)}
  \background{
    \draw[->-=0.5] (g1) to ++(0,0.5);
    \draw[-<-=0.5] (g1) to ++(0,-0.5);
    \draw[->-=0.5] (g2) to ++(0,0.5);
    \draw[-<-=0.5] (g2) to ++(0,-0.5);
  }
}
+
\tikpic{
  \unlabeledinteraction{2}{g}{(0,0)};
  \background{
    \draw[->-=0.5] (g1) to ++(0,0.5);
    \draw[-<-=0.5] (g1) to ++(0,-0.5);
    \draw[->-=0.5] (g2) to ++(-0.25,0.5);
    \draw[-<-=0.5] (g2) to ++(+0.25,0.5);
  }
}
+
\tikpic{
  \unlabeledinteraction{2}{g}{(0,0)}
  \background{
    \draw[->-=0.5] (g1) to ++(0,0.5);
    \draw[-<-=0.5] (g1) to ++(0,-0.5);
    \draw[->-=0.5] (g2) to ++(-0.25,-0.5);
    \draw[-<-=0.5] (g2) to ++(+0.25,-0.5);
  }
}
+
\tikpic{
  \unlabeledinteraction{2}{g}{(0,0)}
  \background{
    \draw[->-=0.5] (g1) to ++(-0.25,0.5);
    \draw[-<-=0.5] (g1) to ++(+0.25,0.5);
    \draw[->-=0.5] (g2) to ++(-0.25,0.5);
    \draw[-<-=0.5] (g2) to ++(+0.25,0.5);
  }
}
+
\tikpic{
  \unlabeledinteraction{2}{g}{(0,0)}
  \background{
    \draw[->-=0.5] (g1) to ++(0,+0.5);
    \draw[-<-=0.5] (g1) to ++(0,-0.5);
    \draw[-<-=0.5] (g2) to ++(0,+0.5);
    \draw[->-=0.5] (g2) to ++(0,-0.5);
  }
}
\\&&&&&\ +
\tikpic{
  \unlabeledinteraction{2}{g}{(0,0)}
  \background{
    \draw[->-=0.5] (g1) to ++(-0.25,-0.5);
    \draw[-<-=0.5] (g1) to ++(+0.25,-0.5);
    \draw[->-=0.5] (g2) to ++(-0.25,-0.5);
    \draw[-<-=0.5] (g2) to ++(+0.25,-0.5);
  }
}
+
\tikpic{
  \unlabeledinteraction{2}{g}{(0,0)}
  \background{
    \draw[-<-=0.5] (g1) to ++(0,0.5);
    \draw[->-=0.5] (g1) to ++(0,-0.5);
    \draw[->-=0.5] (g2) to ++(-0.25,0.5);
    \draw[-<-=0.5] (g2) to ++(+0.25,0.5);
  }
}
+
\tikpic{
  \unlabeledinteraction{2}{g}{(0,0)}
  \background{
    \draw[-<-=0.5] (g1) to ++(0,0.5);
    \draw[->-=0.5] (g1) to ++(0,-0.5);
    \draw[->-=0.5] (g2) to ++(-0.25,-0.5);
    \draw[-<-=0.5] (g2) to ++(+0.25,-0.5);
  }
}
+
\tikpic{
  \unlabeledinteraction{2}{g}{(0,0)}
  \background{
    \draw[-<-=0.5] (g1) to ++(0,0.5);
    \draw[->-=0.5] (g1) to ++(0,-0.5);
    \draw[-<-=0.5] (g2) to ++(0,0.5);
    \draw[->-=0.5] (g2) to ++(0,-0.5);
  }
}
\end{align}
Where the diagrams with open-circle interaction points are
\begin{align}
&&
\tikpic[0pt]{
  \oneelinteraction[white]{f}{$\times$}{(-0.75,0)}{(0,0)};
  \background{
    \draw[->-=0.5] (f1) to ++(0,+0.5);
    \draw[-<-=0.5] (f1) to ++(0,-0.5);
  }
}
\equiv
  f_p^q
  \tl{a}_q^p
&&
\tikpic[0pt]{
  \unlabeledinteraction[white]{2}{g}{(0,0)};
  \background{
    \draw[->-=0.5] (g1) to ++(0,+0.5);
    \draw[-<-=0.5] (g1) to ++(0,-0.5);
    \draw[->-=0.5] (g2) to ++(0,+0.5);
    \draw[-<-=0.5] (g2) to ++(0,-0.5);
  }
}
\equiv
  \tfrac{1}{4}
  \ol{g}_{pq}^{rs}
  \tl{a}_{rs}^{pq}
\end{align}
and those with closed-circle interaction points are their expansions in terms of quasiparticle operators $\{b_p\}=\{a_i\dg\}\cup\{a_a\}$ and $\{b_p\dg\}=\{a_i\}\cup\{a_a\dg\}$.
In terms of KM notation, using the original set of operators, these diagram expansions correspond to the following equations.
\begin{align}
  f_p^q\tl{a}_q^p
=&\
  f_a^b\tl{a}_b^a
+
  f_a^i\tl{a}_i^a
+
  f_i^a\tl{a}_a^i
+
  f_i^j\tl{a}_j^i
\\
  \tfrac{1}{4}\ol{g}_{pq}^{rs}\tl{a}_{rs}^{pq}
=&\
  \tfrac{1}{4}
  \ol{g}_{ab}^{cd}\tl{a}_{cd}^{ab}
+
  \tfrac{1}{2}
  \ol{g}_{ab}^{ci}\tl{a}_{ci}^{ab}
+
  \tfrac{1}{2}
  \ol{g}_{ai}^{bc}\tl{a}_{bc}^{ai}
+
  \tfrac{1}{4}
  \ol{g}_{ab}^{ij}\tl{a}_{ij}^{ab}
+
  \ol{g}_{ai}^{bj}\tl{a}_{bj}^{ai}
+
  \tfrac{1}{4}
  \ol{g}_{ij}^{ab}\tl{a}_{ab}^{ij}
+
  \tfrac{1}{2}
  \ol{g}_{ia}^{jk}\tl{a}_{jk}^{ia}
+
  \tfrac{1}{2}
  \ol{g}_{ij}^{ka}\tl{a}_{ka}^{ij}
+
  \tfrac{1}{4}\ol{g}_{ij}^{kl}\tl{a}_{kl}^{ij}
\end{align}
The CI double excitation operator is given by
\begin{align}
&&
  \op{C}_2
=
  \tfrac{1}{4}
  \sum_{ijab}
  c_{ab}^{ij}
  b_a\dg b_b\dg b_j\dg b_i\dg
=
\tikpic{
  \amplitudeinteraction{2}{c}{(0,-0.25)};
  \background{
    \draw[->-=0.5] (c1) to ++(-0.25,0.5);
    \draw[-<-=0.5] (c1) to ++(+0.25,0.5);
    \draw[->-=0.5] (c2) to ++(-0.25,0.5);
    \draw[-<-=0.5] (c2) to ++(+0.25,0.5);
  }
}
&&
\pr{=\tfrac{1}{4}c_{ab}^{ij}a_{ij}^{ab}=\tfrac{1}{4}c_{ab}^{ij}\tl{a}_{ij}^{ab}\ \text{in KM notation}}\ .
\end{align}
The sigma vector elements are therefore given by
\begin{align}
  \si_0
=
  \ip{\F|H_c(c_0+\op{C}_2)|\F}
=&\
c_0
\pr{
\tikpic{
  \oneelinteraction[white]{f}{$\times$}{(-0.25,0.25)}{(0.5,0.25)};
  \background{
    \draw[->-=0.5] (f1) to ++(+0.00,+0.5);
    \draw[-<-=0.5] (f1) to ++(+0.00,-0.5);
  }
  \draw[white] (-0.25,-0.75) to (0.5,-0.75);
  \padborder{5pt}
  \draw[thick,double] (current bounding box.north west) to
        (current bounding box.north east);
  \draw[thick,double] (current bounding box.south west) to
        (current bounding box.south east);
}
+
\tikpic{
  \unlabeledinteraction[white]{2}{g}{(0,0.25)};
  \background{
    \draw[->-=0.5] (g1) to ++(+0.00,+0.5);
    \draw[-<-=0.5] (g1) to ++(+0.00,-0.5);
    \draw[->-=0.5] (g2) to ++(+0.00,+0.5);
    \draw[-<-=0.5] (g2) to ++(+0.00,-0.5);
  }
  \draw[white] (0.0,-0.75) to (1.0,-0.75);
  \padborder{5pt}
  \draw[thick,double] (current bounding box.north west) to
        (current bounding box.north east);
  \draw[thick,double] (current bounding box.south west) to
        (current bounding box.south east);
}
}
+
\tikpic{
  \oneelinteraction[white]{f}{$\times$}{(-0.25,0.25)}{(0.5,0.25)};
  \amplitudeinteraction{2}{c}{(0,-0.75)};
  \background{
    \draw[->-=0.5] (f1) to ++(+0.00,+0.5);
    \draw[-<-=0.5] (f1) to ++(+0.00,-0.5);
    \draw[->-=0.5] (c1) to ++(-0.25,+0.5);
    \draw[-<-=0.5] (c1) to ++(+0.25,+0.5);
    \draw[->-=0.5] (c2) to ++(-0.25,+0.5);
    \draw[-<-=0.5] (c2) to ++(+0.25,+0.5);
  }
  \padborder{5pt}
  \draw[thick,double] (current bounding box.north west) to
        (current bounding box.north east);
  \draw[thick,double] (current bounding box.south west) to
        (current bounding box.south east);
}
+
\tikpic{
  \unlabeledinteraction[white]{2}{g}{(0,0.25)};
  \amplitudeinteraction{2}{c}{(0,-0.75)};
  \background{
    \draw[->-=0.5] (g1) to ++(+0.00,+0.5);
    \draw[-<-=0.5] (g1) to ++(+0.00,-0.5);
    \draw[->-=0.5] (g2) to ++(+0.00,+0.5);
    \draw[-<-=0.5] (g2) to ++(+0.00,-0.5);
    \draw[->-=0.5] (c1) to ++(-0.25,+0.5);
    \draw[-<-=0.5] (c1) to ++(+0.25,+0.5);
    \draw[->-=0.5] (c2) to ++(-0.25,+0.5);
    \draw[-<-=0.5] (c2) to ++(+0.25,+0.5);
  }
  \padborder{5pt}
  \draw[thick,double] (current bounding box.north west) to
        (current bounding box.north east);
  \draw[thick,double] (current bounding box.south west) to
        (current bounding box.south east);
}
\\
  \si_{ab}^{ij}
=
  \ip{\F_{ij}^{ab}|H_c(c_0+\op{C}_2)|\F}
=&\
c_0
\pr{
\tikpic{
  \oneelinteraction[white]{f}{$\times$}{(-0.75,0)}{(0,0)};
  \interactionpoint{a}{(-0.75,1.0)};
  \interactionpoint{i}{(-0.25,1.0)};
  \interactionpoint{b}{(+0.25,1.0)};
  \interactionpoint{j}{(+0.75,1.0)};
  \node[above=0.5pt of i] () {\small{$i$}};
  \node[above=0.5pt of a] () {\small{$a$}};
  \node[above=0.5pt of j] () {\small{$j$}};
  \node[above=0.5pt of b] () {\small{$b$}};
  \background{
    \draw[->-=0.5] (f1) to ++(0,+0.5);
    \draw[-<-=0.5] (f1) to ++(0,-0.5);
    \draw[-<-=0.5] (a)  to ++(0,-0.5);
    \draw[->-=0.5] (i)  to ++(0,-0.5);
    \draw[-<-=0.5] (b)  to ++(0,-0.5);
    \draw[->-=0.5] (j)  to ++(0,-0.5);
  }
  \draw[white] (-0.5,-1.0) to (+0.5,-1.0);
  \padborder{5pt};
  \draw[thick,double] (current bounding box.north west) to
        (current bounding box.north east);
  \draw[thick,double] (current bounding box.south west) to
        (current bounding box.south east);
}
+
\tikpic{
  \unlabeledinteraction[white]{2}{g}{(-0.5,0)};
  \interactionpoint{a}{(-0.75,1.0)};
  \interactionpoint{i}{(-0.25,1.0)};
  \interactionpoint{b}{(+0.25,1.0)};
  \interactionpoint{j}{(+0.75,1.0)};
  \node[above=0.5pt of i] () {\small{$i$}};
  \node[above=0.5pt of a] () {\small{$a$}};
  \node[above=0.5pt of j] () {\small{$j$}};
  \node[above=0.5pt of b] () {\small{$b$}};
  \background{
    \draw[->-=0.5] (g1) to ++(0,+0.5);
    \draw[-<-=0.5] (g1) to ++(0,-0.5);
    \draw[->-=0.5] (g2) to ++(0,+0.5);
    \draw[-<-=0.5] (g2) to ++(0,-0.5);
    \draw[-<-=0.5] (a)  to ++(0,-0.5);
    \draw[->-=0.5] (i)  to ++(0,-0.5);
    \draw[-<-=0.5] (b)  to ++(0,-0.5);
    \draw[->-=0.5] (j)  to ++(0,-0.5);
  }
  \draw[white] (-0.5,-1.0) to (+0.5,-1.0);
  \padborder{5pt};
  \draw[thick,double] (current bounding box.north west) to
        (current bounding box.north east);
  \draw[thick,double] (current bounding box.south west) to
        (current bounding box.south east);
}}
+
\tikpic{
  \oneelinteraction[white]{f}{$\times$}{(-0.75,0)}{(0,0)};
  \interactionpoint{a}{(-0.75,1.0)};
  \interactionpoint{i}{(-0.25,1.0)};
  \interactionpoint{b}{(+0.25,1.0)};
  \interactionpoint{j}{(+0.75,1.0)};
  \node[above=0.5pt of i] () {\small{$i$}};
  \node[above=0.5pt of a] () {\small{$a$}};
  \node[above=0.5pt of j] () {\small{$j$}};
  \node[above=0.5pt of b] () {\small{$b$}};
  \amplitudeinteraction{2}{c}{(-0.5,-1.0)};
  \background{
    \draw[->-=0.5] (f1) to ++(0,+0.5);
    \draw[-<-=0.5] (f1) to ++(0,-0.5);
    \draw[-<-=0.5] (a)  to ++(0,-0.5);
    \draw[->-=0.5] (i)  to ++(0,-0.5);
    \draw[-<-=0.5] (b)  to ++(0,-0.5);
    \draw[->-=0.5] (j)  to ++(0,-0.5);
    \draw[->-=0.5] (c1) to ++(-0.25,+0.5);
    \draw[-<-=0.5] (c1) to ++(+0.25,+0.5);
    \draw[->-=0.5] (c2) to ++(-0.25,+0.5);
    \draw[-<-=0.5] (c2) to ++(+0.25,+0.5);
  }
  \padborder{5pt};
  \draw[thick,double] (current bounding box.north west) to
        (current bounding box.north east);
  \draw[thick,double] (current bounding box.south west) to
        (current bounding box.south east);
}
+
\tikpic{
  \unlabeledinteraction[white]{2}{g}{(-0.5,0)};
  \interactionpoint{a}{(-0.75,1.0)};
  \interactionpoint{i}{(-0.25,1.0)};
  \interactionpoint{b}{(+0.25,1.0)};
  \interactionpoint{j}{(+0.75,1.0)};
  \node[above=0.5pt of i] () {\small{$i$}};
  \node[above=0.5pt of a] () {\small{$a$}};
  \node[above=0.5pt of j] () {\small{$j$}};
  \node[above=0.5pt of b] () {\small{$b$}};
  \amplitudeinteraction{2}{c}{(-0.5,-1.0)};
  \background{
    \draw[->-=0.5] (g1) to ++(0,+0.5);
    \draw[-<-=0.5] (g1) to ++(0,-0.5);
    \draw[->-=0.5] (g2) to ++(0,+0.5);
    \draw[-<-=0.5] (g2) to ++(0,-0.5);
    \draw[-<-=0.5] (a)  to ++(0,-0.5);
    \draw[->-=0.5] (i)  to ++(0,-0.5);
    \draw[-<-=0.5] (b)  to ++(0,-0.5);
    \draw[->-=0.5] (j)  to ++(0,-0.5);
    \draw[->-=0.5] (c1) to ++(-0.25,+0.5);
    \draw[-<-=0.5] (c1) to ++(+0.25,+0.5);
    \draw[->-=0.5] (c2) to ++(-0.25,+0.5);
    \draw[-<-=0.5] (c2) to ++(+0.25,+0.5);
  }
  \padborder{5pt};
  \draw[thick,double] (current bounding box.north west) to
        (current bounding box.north east);
  \draw[thick,double] (current bounding box.south west) to
        (current bounding box.south east);
}
\end{align}
In the $\si_0$ expression, only the final diagram is non-vanishing.
\begin{align}
&&
  \si_0
=&\
\tikpic{
  \unlabeledinteraction[white]{2}{g}{(0,0.25)};
  \amplitudeinteraction{2}{c}{(0,-0.75)};
  \background{
    \draw[->-=0.5] (g1) to ++(+0.00,+0.5);
    \draw[-<-=0.5] (g1) to ++(+0.00,-0.5);
    \draw[->-=0.5] (g2) to ++(+0.00,+0.5);
    \draw[-<-=0.5] (g2) to ++(+0.00,-0.5);
    \draw[->-=0.5] (c1) to ++(-0.25,+0.5);
    \draw[-<-=0.5] (c1) to ++(+0.25,+0.5);
    \draw[->-=0.5] (c2) to ++(-0.25,+0.5);
    \draw[-<-=0.5] (c2) to ++(+0.25,+0.5);
  }
  \padborder{5pt}
  \draw[thick,double] (current bounding box.north west) to
        (current bounding box.north east);
  \draw[thick,double] (current bounding box.south west) to
        (current bounding box.south east);
}
=
\tikpic{
  \unlabeledinteraction{2}{g}{(0,0.5)};
  \amplitudeinteraction{2}{c}{(0,-0.5)};
  \background{
    \draw[->-=0.5,bend left ] (c1) to (g1);
    \draw[-<-=0.5,bend right] (c1) to (g1);
    \draw[->-=0.5,bend left ] (c2) to (g2);
    \draw[-<-=0.5,bend right] (c2) to (g2);
  }
}
=
  \tfrac{1}{4}
  \ol{g}_{kl}^{cd}c_{cd}^{kl}
\end{align}
In the $\si_{ab}^{ij}$ expression, the first diagram vanishes and the second diagram evaluates as follows.
\begin{align}
&&
\tikpic{
  \unlabeledinteraction{2}{g}{(-0.5,-0.5)};
  \interactionpoint{a}{(-0.75,0.5)};
  \interactionpoint{i}{(-0.25,0.5)};
  \interactionpoint{b}{(+0.25,0.5)};
  \interactionpoint{j}{(+0.75,0.5)};
  \node[above=0.5pt of i] () {\small{$i$}};
  \node[above=0.5pt of a] () {\small{$a$}};
  \node[above=0.5pt of j] () {\small{$j$}};
  \node[above=0.5pt of b] () {\small{$b$}};
  \background{
    \draw[->-=0.5] (g1) to ++(0,+0.5);
    \draw[-<-=0.5] (g1) to ++(0,-0.5);
    \draw[->-=0.5] (g2) to ++(0,+0.5);
    \draw[-<-=0.5] (g2) to ++(0,-0.5);
    \draw[-<-=0.5] (a)  to ++(0,-0.5);
    \draw[->-=0.5] (i)  to ++(0,-0.5);
    \draw[-<-=0.5] (b)  to ++(0,-0.5);
    \draw[->-=0.5] (j)  to ++(0,-0.5);
  }
  \padborder{5pt};
  \draw[thick,double] (current bounding box.north west) to
        (current bounding box.north east);
  \draw[thick,double] (current bounding box.south west) to
        (current bounding box.south east);
}
=
\tikpic{
  \unlabeledinteraction{2}{g}{(-0.5,-0.25)};
  \interactionpoint{a}{(-0.75,0.25)};
  \interactionpoint{i}{(-0.25,0.25)};
  \interactionpoint{b}{(+0.25,0.25)};
  \interactionpoint{j}{(+0.75,0.25)};
  \node[above=0.5pt of i] () {\small{$i$}};
  \node[above=0.5pt of a] () {\small{$a$}};
  \node[above=0.5pt of j] () {\small{$j$}};
  \node[above=0.5pt of b] () {\small{$b$}};
  \background{
    \draw[->-=0.5] (g1) to (a);
    \draw[-<-=0.5] (g1) to (i);
    \draw[->-=0.5] (g2) to (b);
    \draw[-<-=0.5] (g2) to (j);
  }
}
=
  \ol{g}_{ab}^{ij}
\end{align}


\end{document}
