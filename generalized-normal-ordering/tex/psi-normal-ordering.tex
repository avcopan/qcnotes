\documentclass[11pt,fleqn]{article}
\usepackage[cm]{fullpage}
\usepackage{mathtools} %includes amsmath
\usepackage{amsfonts}
\usepackage{bm}
\usepackage{xfrac}
\usepackage{url}
%greek letters
\renewcommand{\a}{\alpha}    %alpha
\renewcommand{\b}{\beta}     %beta
\newcommand{\g}{\gamma}      %gamma
\newcommand{\G}{\Gamma}      %Gamma
\renewcommand{\d}{\delta}    %delta
\newcommand{\D}{\Delta}      %Delta
\newcommand{\e}{\varepsilon} %epsilon
\newcommand{\ev}{\epsilon}   %epsilon*
\newcommand{\z}{\zeta}       %zeta
\newcommand{\h}{\eta}        %eta
\renewcommand{\th}{\theta}   %theta
\newcommand{\Th}{\Theta}     %Theta
\newcommand{\io}{\iota}      %iota
\renewcommand{\k}{\kappa}    %kappa
\newcommand{\la}{\lambda}    %lambda
\newcommand{\La}{\Lambda}    %Lambda
\newcommand{\m}{\mu}         %mu
\newcommand{\n}{\nu}         %nu %xi %Xi %pi %Pi
\newcommand{\p}{\rho}        %rho
\newcommand{\si}{\sigma}     %sigma
\newcommand{\siv}{\varsigma} %sigma*
\newcommand{\Si}{\Sigma}     %Sigma
\renewcommand{\t}{\tau}      %tau
\newcommand{\up}{\upsilon}   %upsilon
\newcommand{\f}{\phi}        %phi
\newcommand{\F}{\Phi}        %Phi
\newcommand{\x}{\chi}        %chi
\newcommand{\y}{\psi}        %psi
\newcommand{\Y}{\Psi}        %Psi
\newcommand{\w}{\omega}      %omega
\newcommand{\W}{\Omega}      %Omega
%ornaments
\newcommand{\eth}{\ensuremath{^\text{th}}}
\newcommand{\rst}{\ensuremath{^\text{st}}}
\newcommand{\ond}{\ensuremath{^\text{nd}}}
\newcommand{\ord}[1]{\ensuremath{^{(#1)}}}
\newcommand{\dg}{\ensuremath{^\dagger}}
\newcommand{\bigo}{\ensuremath{\mathcal{O}}}
\newcommand{\tl}{\ensuremath{\tilde}}
\newcommand{\ol}[1]{\ensuremath{\overline{#1}}}
\newcommand{\ul}[1]{\ensuremath{\underline{#1}}}
\newcommand{\op}[1]{\ensuremath{\hat{#1}}}
\newcommand{\ot}{\ensuremath{\otimes}}
\newcommand{\wg}{\ensuremath{\wedge}}
%text
\newcommand{\tr}{\ensuremath{\hspace{1pt}\mathrm{tr}\hspace{1pt}}}
\newcommand{\Alt}{\ensuremath{\mathrm{Alt}}}
\newcommand{\sgn}{\ensuremath{\mathrm{sgn}}}
\newcommand{\occ}{\ensuremath{\mathrm{occ}}}
\newcommand{\vir}{\ensuremath{\mathrm{vir}}}
\newcommand{\spn}{\ensuremath{\mathrm{span}}}
\newcommand{\vac}{\ensuremath{\mathrm{vac}}}
\newcommand{\bs}{\ensuremath{\text{\textbackslash}}}
\newcommand{\im}{\ensuremath{\mathrm{im}\hspace{1pt}}}
\renewcommand{\sp}{\hspace{30pt}}
%dots
\newcommand{\ld}{\ensuremath{\ldots}}
\newcommand{\cd}{\ensuremath{\cdots}}
\newcommand{\vd}{\ensuremath{\vdots}}
\newcommand{\dd}{\ensuremath{\ddots}}
\newcommand{\etc}{\ensuremath{\mathinner{\mkern-1mu\cdotp\mkern-2mu\cdotp\mkern-2mu\cdotp\mkern-1mu}}}
%fonts
\newcommand{\bmit}[1]{{\bfseries\itshape\mathversion{bold}#1}}
\newcommand{\mc}[1]{\ensuremath{\mathcal{#1}}}
\newcommand{\mb}[1]{\ensuremath{\mathbb{#1}}}
\newcommand{\mf}[1]{\ensuremath{\mathfrak{#1}}}
\newcommand{\mr}[1]{\ensuremath{\mathrm{#1}}}
\newcommand{\bo}[1]{\ensuremath{\mathbf{#1}}}
%styles
\newcommand{\ts}{\textstyle}
\newcommand{\ds}{\displaystyle}
\newcommand{\phsub}{\ensuremath{_{\phantom{p}}}}
\newcommand{\phsup}{\ensuremath{^{\phantom{p}}}}
%fractions, derivatives, parentheses, brackets, etc.
\newcommand{\pr}[1]{\ensuremath{\left(#1\right)}}
\newcommand{\brk}[1]{\ensuremath{\left[#1\right]}}
\newcommand{\fr}[2]{\ensuremath{\dfrac{#1}{#2}}}
\newcommand{\pd}[2]{\ensuremath{\frac{\partial#1}{\partial#2}}}
\newcommand{\pt}{\ensuremath{\partial}}
\newcommand{\br}[1]{\ensuremath{\langle#1|}}
\newcommand{\kt}[1]{\ensuremath{|#1\rangle}}
\newcommand{\ip}[1]{\ensuremath{\langle#1\rangle}}
\newcommand{\NO}[1]{\ensuremath{{\bm{:}}#1{}{\bm{:}}}}
\newcommand{\floor}[1]{\ensuremath{\left\lfloor#1\right\rfloor}}
\newcommand{\ceil}[1]{\ensuremath{\left\lceil#1\right\rceil}}
\usepackage{stackengine}
\newcommand{\GNO}[1]{\setstackgap{S}{0.7pt}\ensuremath{\Shortstack{\textbf{.} \textbf{.} \textbf{.}}#1\Shortstack{\textbf{.} \textbf{.} \textbf{.}}}}
\newcommand{\cmtr}[2]{\ensuremath{[\cdot,#2]^{#1}}}
\newcommand{\cmtl}[2]{\ensuremath{[#2,\cdot]^{#1}}}
%structures
\newcommand{\eqn}[1]{(\ref{#1})}
\newcommand{\ma}[1]{\ensuremath{\begin{bmatrix}#1\end{bmatrix}}}
\newcommand{\ar}[1]{\ensuremath{\begin{matrix}#1\end{matrix}}}
\newcommand{\miniar}[1]{\ensuremath{\begin{smallmatrix}#1\end{smallmatrix}}}
%contractions
\usepackage{simplewick}
\usepackage[nomessages]{fp}
\newcommand{\ctr}[6][0]{\FPeval\height{0.6+#1*0.5}\ensuremath{\contraction[\height ex]{#2}{#3}{#4}{#5}}}
\newcommand{\ccr}[4]{\ctr[0.7]{#1}{#2}{#3}{#4}{}\ctr{#1}{#2}{#3}{#4}}
\usepackage{calc}
\makeatletter
\def\@hspace#1{\begingroup\setlength\dimen@{#1}\hskip\dimen@\endgroup}
\makeatother
\newcommand{\halfphantom}[1]{\hspace{\widthof{#1}*\real{0.5}}}
\newcommand{\fullctr}[1]{\ensuremath{\contraction[0.5ex]{}{\vphantom{#1}}{\hphantom{#1}}{}{}{}\contraction[0.5ex]{}{\vphantom{#1}}{\halfphantom{#1}}{}{}{}#1}}
%math sections
\usepackage{cleveref}
\usepackage{amsthm}
\usepackage{thmtools}
\declaretheoremstyle[spaceabove=10pt,spacebelow=10pt,bodyfont=\small]{mystyle}
\theoremstyle{mystyle}
\newtheorem{dfn}{Definition}[section]
\crefname{dfn}{definition}{definitions}
\Crefname{dfn}{Def}{Defs}
\newtheorem{thm}{Theorem}[section]
\crefname{thm}{theorem}{theorems}
\Crefname{thm}{Thm}{Thms}
\newtheorem{cor}{Corollary}[section]
\crefname{cor}{corollary}{corollaries}
\Crefname{cor}{Cor}{Cors}
\newtheorem{lem}{Lemma}[section]
\crefname{lem}{lemma}{lemmas}
\Crefname{lem}{Lem}{Lems}
\newtheorem{rmk}{Remark}[section]
\crefname{rmk}{remark}{remarks}
\Crefname{rmk}{Rmk}{Rmks}
\newtheorem{pro}{Proposition}[section]
\crefname{pro}{proposition}{propositions}
\Crefname{pro}{Prop}{Props}
\newtheorem{ntt}{Notation}[section]
\crefname{ntt}{notation}{notations}
\Crefname{ntt}{Notation}{Notations}
\newcommand{\logbox}{\ensuremath{\text{\makebox[\widthof{exp}][c]{ln}}}}
\newcommand{\expbox}{\ensuremath{\text{\makebox[\widthof{exp}][c]{exp}}}}

\numberwithin{equation}{section}
\usepackage{cancel}
\usepackage{scrextend}

\newcommand{\hole}{\circ}
\newcommand{\ptcl}{\bullet}
\newcommand{\circled}[1]{\raisebox{.5pt}{\textcircled{\raisebox{-.9pt}{#1}}}}



\author{Andreas V. Copan}
\date{}
\title{Fock space algebraic methods:\\Normal ordering with respect to $\Y$}

%%%DOCUMENT%%%
\begin{document}

\maketitle

\section{Moments and cumulants}

\subsection{Moments and cumulants in classical statistics}

\begin{dfn}
\label{moments-dfn}
\bmit{Moments of commuting random variables.}
Given a set $\{q_i\}$ of commuting random variables, their \textit{moments} $\g(q_{i_1}\cd q_{i_n})$ are defined as
\begin{align}
\label{moments}
&&
  \g(q_{i_1}\cd q_{i_n})
=
  \ip{q_{i_1}\cd q_{i_n}}
\end{align}
where $\ip{X}$ represents an expecation value.
\end{dfn}

\begin{rmk}
\label{classical-moment-expansion}
\bmit{Expectation values from moments.}
Given a complete set of moments $\{\g(q_{i_1}\cd q_{i_n})\}$, the expectation value of any analytic function $f(q_i)$ of the random variables can be obtained as
\begin{align}
&&
  \ip{f(q_i)}
=
  f(0)
+
  \sum_n
  \fr{1}{n!}
  \sum_{i_1\cd i_n}
  \left.
  \pd{^n f}{q_{i_1}\cd \pt q_{i_n}}
  \right|_{q_i=0}
  \g(q_{i_1}\cd q_{i_n})
\end{align}
which results from expanding $f(q_i)$ in a Taylor series, noting that the expectation value is linear, and applying \Cref{moments-dfn}.
Here and below, sums $\sum_n$ without specified ranges should be taken to run from $1$ to $\infty$.
\end{rmk}

\begin{dfn}
\bmit{Moment and cumulant generating functions of commuting random variables.}
Given a set $\{q_i\}$ of commuting random variables, its \textit{moment-generating function}, $M(\bm\a)$, and its \textit{cumulant-generating function}, $K(\bm\a)$, are defined as
\begin{align}
\label{classical-mgf}
&
  M(\bm\a)
\equiv
  \ip{e^{\sum_i\a_i q_i}}
&&
  M(\bm\a)
=
  1
+
  \sum_n
  \sum_{i_1\cd i_n}
  \fr{\a_{i_1}\cd \a_{i_n}}{n!}
  \g(q_{i_1}\cd q_{i_n})
\\
\label{classical-cgf}
&
  K(\bm\a)
\equiv
  \ln M(\bm\a)
=
  \ln \ip{e^{\sum_i\a_i q_i}}
&&
  K(\bm\a)
=
  0
+
  \sum_n
  \sum_{i_1\cd i_n}
  \fr{\a_{i_1}\cd \a_{i_n}}{n!}
  \la(q_{i_1}\cd q_{i_n})
\end{align}
where $\g(q_{i_1}\cd q_{i_n})$ and $\la(q_{i_1}\cd q_{i_n})$ are (respectively) the \textit{moments} and \textit{cumulants} that they generate.
The moments and cumulants are obtained from $M(\bm\a)$ and $K(\bm\a)$ via
\begin{align}
&
  \g(q_{i_1}\cd q_{i_n})
\equiv
  \left.
  \pd{^n M(\bm\a)}{\a_{i_1}\cd\pt\a_{i_n}}
  \right|_{\bm\a=0}
&&
  \la(q_{i_1}\cd q_{i_n})
\equiv
  \left.
  \pd{^n K(\bm\a)}{\a_{i_1}\cd\pt\a_{i_n}}
  \right|_{\bm\a=0}
\end{align}
as can be seen from equations \ref{classical-mgf} and \ref{classical-cgf}.
\end{dfn}

\begin{pro}
\label{moment-cumulant-relations-classical}
\bmit{Moment-cumulant relations of commuting random variables.}
\textit{The moments and cumulants of a set $\{q_i\}$ of commuting random variables are related via
\begin{align}
&
  \g(q_{i_1}\cd q_{i_n})
=
  \sum_{k=1}^n
  \sum_{(Q_1\etc Q_k)}^{\mc{P}_k(q_{i_1}\etc q_{i_n})}
  \la(Q_1)\etc\la(Q_k)
\\
&
  \la(q_{i_1}\cd q_{i_n})
=
  \sum_{k=1}^n
  (-)^{k+1}
  (k-1)!
  \sum_{(Q_1\etc Q_k)}^{\mc{P}_k(q_{i_1}\etc q_{i_n})}
  \g(Q_1)\etc\g(Q_k)
\end{align}
where $(Q_1,\ld,Q_k)\in\mc{P}_k(q_{i_1}\cd q_{i_n})$ are unique $k$-tuple partitions of the product $q_{i_1}\cd q_{i_n}$.}\footnote{These ``product partitions'' are simply set partitions with each set of operators mapped to their product.}
\begin{addmargin}[1em]{0em}
Proof: See \Cref{moment-cumulant-relations-general}.
\end{addmargin}
\end{pro}



\subsection{Moments and cumulants of particle-hole operators}

\begin{rmk}
\bmit{Motivating the form of $M(\bm\a)$ for particle-hole operators.}
Just as the expectation value of a function of classical random variables can be obtained from its moment expansion (see \Cref{classical-moment-expansion}), the expectation value of the electronic Hamiltonian can be obtained from $\g(a_q^p)$ and $\g(a_{rs}^{pq})$ as
\begin{align*}
&&
  E_e
=
  \ip{H_e(a^p,a_q)}
=
  h_p^q \g(a_q^p)
+
  \tfrac{1}{4}
  g_{pq}^{rs} \g(a_{rs}^{pq})\ .
\end{align*}
where the expectation value is $\ip{\cdot}=\ip{\Y|\cdot|\Y}$, the random variables are $\{q_i\}=\{a_p\}\cup\{a^p\}$, and the moments are $\g(q_{i_1}\cd q_{i_n})=\ip{\Y|q_{i_1}\cd q_{i_n}|\Y}$.
However, the derivatives of $\ip{\Y|e^{\sum_i\a_iq_i}|\Y}$ do not generate $\g(q_1\cd q_n)$ because the $q_i$ do not commute.
Noting that the operator products defining $H_e$ are in \vac-normal order, we can generate the moments we need by normal-ordering the exponential in the moment generating function.
\begin{align}
\label{moment-motivation}
&&
  M(\bm\a)
\equiv
  \ip{\Y|\NO{e^{\sum_i\a_iq_i}}|\Y}
=
  1
+
  \sum_n\sum_{i_1\cd i_n}\frac{\a_{i_1}\cd\a_{i_n}}{n!}
  \ip{\Y|\NO{q_{i_1}\cd a_{i_n}}|\Y}
\end{align}
However, the relation
$
  \pd{^nM(\bm\a)}{\a_{i_1}\cd\pt\a_{i_n}}
=
  \ip{\Y|\NO{q_{i_1}\cd q_{i_n}}|\Y}
$
still does not hold if we take $\bm\a$ to consist of ordinary $\mb{C}$-numbers.
Instead, the fact that $\NO{q_{i_1}\cd q_{i_n}}$ is antisymmetric under index permutation requires that the probe variable derivatives anticommute:
$
  \pd{^nM(\bm\a)}{\a_{i_1}\cd\pt\a_{i_n}}
=
  \e_\pi\pd{^nM(\bm\a)}{\a_{i_{\pi(1)}}\cd\pt\a_{i_{\pi(n)}}}
$.
This is a known property of the so-called \textit{Grassmann numbers} used in Fermion counting statistics.   (See Cahill and Glauber, \textit{Phys.~Rev.~A}, \textbf{59}, 1538 (1999) for more details).
\end{rmk}

\begin{dfn}
\bmit{Grassmann numbers.}
A system of particle-hole operators $\{q_i\}$ can be associated with a set of \textit{Grassmann numbers} which are defined to satisfy $[\a_i,\a_j]_+=[\a_i,q_j]_+=0$, i.e. they anticommute among themselves and with all particle-hole operators.
Consistency demands that derivatives with respect to Grassmann-valued variables also anticommute, i.e. $[\pd{}{\a_i},\pd{}{\a_j}]_+=0$, so that $\pd{}{\a_i}\pd{}{\a_j}\a_j\a_i=\pd{}{\a_i}\a_i=1$.
\end{dfn}

\begin{dfn}
\bmit{Density moment and density cumulant generating functions.}
The \textit{moment-generating function}, $M(\bm\a)$, and \textit{cumulant-generating function}, $K(\bm\a)$, of the particle-hole operators is given by
\begin{align}
&
  M(\bm\a)
\equiv
  \ip{\Y|\NO{e^{\sum_i\a_i q_i}}|\Y}
&&
  M(\bm\a)
=
  1
+
  \sum_n
  \sum_{i_1\cd i_n}
  \fr{\a_{i_1}\cd \a_{i_n}}{n!}
  \g(q_{i_1}\cd q_{i_n})
\\
&
  K(\bm\a)
\equiv
  \ln M(\bm\a)
=
  \ln \ip{\Y|\NO{e^{\sum_i\a_i q_i}}|\Y}
&&
  K(\bm\a)
=
  0
+
  \sum_n
  \sum_{i_1\cd i_n}
  \fr{\a_{i_1}\cd \a_{i_n}}{n!}
  \la(q_{i_1}\cd q_{i_n})
\end{align}
where $\bm\a$ consists of Grassmann variables.
The \textit{moments} and \textit{cumulants} are obtained from their generating functions as 
\begin{align}
&
  \g(q_{i_1}\cd q_{i_n})
\equiv
  \left.
  \pd{^n M(\bm\a)}{\a_{i_n}\cd\pt\a_{i_1}}
  \right|_{\bm\a=0}
&&
  \la(q_{i_1}\cd q_{i_n})
\equiv
  \left.
  \pd{^n K(\bm\a)}{\a_{i_n}\cd\pt\a_{i_1}}
  \right|_{\bm\a=0}
\end{align}
These are sometimes called \textit{density moments} and \textit{density cumulants} since they characterize the distribution of the electronic density matrix, $\kt{\Y}\br{\Y}$.
\end{dfn}

\begin{pro}
\label{moment-cumulant-relations-quantum}
\bmit{Density moment-cumulant relations.}
\textit{The density moments and cumulants are related via
\begin{align}
\label{moments-from-cumulants-quantum}
&
  \g(q_{i_1}\cd q_{i_n})
=
  \sum_{k=1}^n
  \sum_{(Q_1\etc Q_k)}^{\mc{P}_k(q_{i_1}\etc q_{i_n})}
  (-)^{n_{\bo{Q}}}
  \la(Q_1)\etc\la(Q_k)
\\
\label{cumulants-from-moments-quantum}
&
  \la(q_{i_1}\cd q_{i_n})
=
  \sum_{k=1}^n
  (-)^{k+1}
  (k-1)!
  \sum_{(Q_1\etc Q_k)}^{\mc{P}_k(q_{i_1}\etc q_{i_n})}
  (-)^{n_{\bo{Q}}}
  \g(Q_1)\etc\g(Q_k)
\end{align}
where $(Q_1,\ld,Q_k)\in\mc{P}_k(q_{i_1}\cd q_{i_n})$ are unique $k$-tuple partitions of the product $q_{i_1}\cd q_{i_n}$.
The operators within each block $Q_i$ of the partition are taken to appear in the same order as they do in the original product, and $n_{\bo{Q}}$ is the number of transpositions required to turn $q_{i_1}\cd q_{i_n}$ into $Q_1\cd Q_k$.}
\begin{addmargin}[1em]{0em}
Proof: See \Cref{moment-cumulant-relations-general}.
\end{addmargin}
\end{pro}


\section{$\Y$-normal ordering}

\begin{dfn}
\bmit{Generalized contractions.}
\textit{Generalized contractions} are scalar functions of the particle-hole operators
\begin{align*}
&&
\{
  \ol{q}_{i_1},\
  \ctr{}{q}{_{i_1}}{q}{_{i_2}}
         q_{i_1}q_{i_2},\
  \ctr{}{q}{_{i_1}}{q}{_{i_2}q_{i_3}}
  \ctr{}{q}{_{i_1}q_{i_2}}{q}{_{i_3}}
         q_{i_1}q_{i_2}q_{i_3},\
  \ctr{}{q}{_{i_1}}{q}{_{i_2}q_{i_3}q_{i_4}}
  \ctr{}{q}{_{i_1}q_{i_2}}{q}{_{i_3}q_{i_4}}
  \ctr{}{q}{_{i_1}q_{i_2}q_{i_3}}{q}{_{i_4}}
         q_{i_1}q_{i_2}q_{i_3}q_{i_4},\ld
\}
\end{align*}
i.e. each $n$-tuple contraction
$ \ctr{}{q}{_{i_1}}{q}{_{i_2}\etc q_{i_n}}
  \ctr{}{q}{_{i_1}q_{i_2}\etc}{q}{_{i_n}}
  q_{i_1}q_{i_2}\etc q_{i_n}$
associates a scalar value with the ordered list of operators $(q_{i_1}, q_{i_2}, \ld, q_{i_n})$.
\end{dfn}

\begin{dfn}
\label{gno}
\bmit{$\Y$-normal order and $\Y$-normal contractions.}
The \textit{$\Y$-normal order} for particle-hole operator strings $Q$ is defined as
\begin{align*}
&&
  \GNO{Q}
\equiv
  Q
-
  \GNO{\ol{Q}}
\end{align*}
where the generalized contractions are chosen such that $\ip{\Y|\GNO{q_1\etc q_n}|\Y}=0$ for all $n$.
The sets of generalized contractions satisfying these conditions are called \textit{$\Y$-normal contractions}.
\end{dfn}

\begin{pro}
\label{odd-contractions-vanish}
\bmit{Odd contractions vanish.}
\textit{For odd $n$, $\ctr{}{q}{_{i_1}}{q}{_{i_2}\etc q_{i_n}}\ctr{}{q}{_{i_1}q_{i_2}\etc}{q}{_{i_n}}q_{i_1}q_{i_2}\etc q_{i_n}=0$ if this is a $\Y$-normal contraction and $\Y$ has a fixed particle number.}
\begin{addmargin}[1em]{0em}
Proof:
Applying \Cref{wick-vacuum-expectation} to a single particle-hole operator, we have $\ol{q}=\ip{\Y|q|\Y}=0$ since $\Y$ has a definite particle number.\footnote{Each $\mb{A}(\mc{H}^{\otimes n})\subset F(\mc{H})$ forms an orthogonal subspace of $F(\mc{H})$.}
Now, assume the conclusion holds for odd $n$ and consider the $(n+2)$-tuple contraction $\ctr{}{q}{_{i_1}}{q}{_{i_2}\etc q_{i_n}}\ctr{}{q}{_{i_1}q_{i_2}\etc}{q}{_{i_n}}q_{i_1}q_{i_2}\etc q_{i_{n+2}}$.
By \Cref{wick-vacuum-expectation} we have $\ip{\Y|q_{i_1}\etc q_{i_{n+2}}|\Y}=\ol{\ol{q_{i_1}\etc q_{i_{n+2}}}}{\ }'+\ctr{}{q}{_{i_1}}{q}{_{i_2}\etc q_{i_n}}\ctr{}{q}{_{i_1}q_{i_2}\etc}{q}{_{i_n}}q_{i_1}q_{i_2}\etc q_{i_{n+2}}$ where the prime indicates that the $(n+2)$-tuple contraction has been separated out.
Since $\Y$ has a definite particle number, we have $\ip{\Y|q_{i_1}\etc q_{i_{n+2}}|\Y}=0$.
Furthermore, every term in $\ol{\ol{q_{i_1}\etc q_{i_{n+2}}}}{\ }'$ must involve an odd contraction of order $n$ or less in order to fully contract the product, which means that this term vanishes as well.
This leaves $\ctr{}{q}{_{i_1}}{q}{_{i_2}\etc q_{i_n}}\ctr{}{q}{_{i_1}q_{i_2}\etc}{q}{_{i_n}}q_{i_1}q_{i_2}\etc q_{i_{n+2}}=0$, so the conclusion holds in general.
\end{addmargin}
\end{pro}


\subsection{Generalized Wick's theorem}

\begin{lem}
\label{psi-expectation-partitions}
\textit{The $\Y$ expectation value of a string $Q$ of particle-hole operators is given by
\begin{align}
&&
  \ip{\Y|Q|\Y}
=
  \sum_{(Q_1\cd Q_k)}^{\mc{P}(Q)}(-)^{n_\bo{Q}}\ \fullctr{Q}_1\etc\fullctr{Q}_k
\end{align}
where $\fullctr{Q}$ represents
$ \ctr{}{q}{_{i_1}}{q}{_{i_2}\etc q_{i_n}}
  \ctr{}{q}{_{i_1}q_{i_2}\etc}{q}{_{i_n}}
  q_{i_1}q_{i_2}\etc q_{i_n}$, a contraction of all of the operators in $Q$, and $\mc{P}(Q)$ are product partitions of $Q$.  $n_\bo{Q}$ is the number of transpositions required to achieve the permutation $Q\mapsto Q_1\cd Q_k$.}
\begin{addmargin}[1em]{0em}
Proof: By \Cref{wick-vacuum-expectation}, we have $\ip{\Y|Q|\Y}=\ol{\ol{Q}}$.
Furthermore, each complete contraction of $Q$ partitions of its operators into disjoint subsets $(Q_1,\ld,Q_k)$ of operators that are contracted with each other.
Finally, disentangling the contracted product is achieved by a permutation which places contraction partners adjacent to each other, i.e. $Q\mapsto Q_1\cd Q_k$.
The signature of such a permutation, $(-)^{n_{\bo{Q}}}$, is unambiguous because odd contractions vanish (see \Cref{odd-contractions-vanish}).
Therefore, $\ip{\Y|Q|\Y}=\ol{\ol{Q}}=\sum_{\bo{Q}}^{\mc{P}(Q)}(-)^{n_\bo{Q}}\ \fullctr{Q}_1\etc\fullctr{Q}_k$.
\end{addmargin}
\end{lem}

\begin{thm}
\label{generalized-wicks-theorem}
\bmit{Generalized Wick's theorem.}
\textit{Any operator $Q$ which is in \vac-normal order $(\NO{})$ can be expanded as
\begin{align}
\label{psi-normal-wick-expansion}
&&
  Q
=
  \GNO{Q}
+
  \GNO{\ol{Q}}
\end{align}
where $\GNO{}$ inticates a $\Y$-normal-ordered product and the generalized $\Y$-normal contractions are cumulants. That is,
\begin{align}
\label{cumulant-contraction}
&&
  \ctr{}{q}{_{i_1}}{q}{_{i_2}\etc q_{i_n}}\ctr{}{q}{_{i_1}q_{i_2}\etc}{q}{_{i_n}}
  q_{i_1}q_{i_2}\etc q_{i_n}
=
  \la(q_{i_1}\cd q_{i_n})
\end{align}
for each $n$-tuple contraction in $\GNO{\ol{Q}}$.}
\begin{addmargin}[1em]{0em}
Proof:
Equation \ref{psi-normal-wick-expansion} is required by \Cref{gno}, so it remains to be proven that the $\Y$-normal contractions are cumulants.
The theorem holds for $n=2$ since
$
  \ip{\Y|q_{i_1}q_{i_2}|\Y}
=
  \ctr{}{q}{_{i_1}}{q}{_{i_2}}
  q_{i_1}q_{i_2}
$
follows from \Cref{psi-expectation-partitions} and
$
  \ip{\Y|q_{i_1}q_{i_2}|\Y}
=
  \g(q_{i_1}q_{i_2})
=
  \la(q_{i_1}q_{i_2})
$.
Now, assume it holds for all contractions up to $n$ and consider $\fullctr{Q}$ for $Q$ of length $n$.
By \Cref{psi-expectation-partitions}, we have
\begin{align*}
  \ip{\Y|Q|\Y}
=
  \g(Q)
=
  \fullctr{Q}
+
  \sum_{k=1}^{n-1}
  \sum_{(Q_1\cd Q_k)}^{\mc{P}_k(Q)}
  (-)^{n_\bo{Q}}\ \fullctr{Q}_1\etc\fullctr{Q}_k
\end{align*}
which gives
\begin{align*}
  \fullctr{Q}
=
  \g(Q)
-
  \sum_{k=1}^{n-1}
  \sum_{(Q_1\cd Q_k)}^{\mc{P}_k(Q)}
  (-)^{n_\bo{Q}}\ \la(Q_1)\cd\la(Q_k)
\end{align*}
which implies $\fullctr{Q}=\la(Q)$ from the moment-cumulant relations (\Cref{moment-cumulant-relations-quantum}, equation \ref{moments-from-cumulants-quantum}).
By induction, the claim holds for all $n$.
\end{addmargin}
\end{thm}



\newpage
\appendix

\section{Derivation of the moment-cumulant relations}

\begin{pro}
\label{moment-cumulant-relations-general}
\bmit{Moment-cumulant relations (general).}
\textit{The moments and cumulants of a set $\{q_i\}$ of random variables are related via
\begin{align}
&
  \g(q_{i_1}\cd q_{i_n})
=
  \sum_{k=1}^n
  \sum_{(Q_1\etc Q_k)}^{\mc{P}_k(q_{i_1}\etc q_{i_n})}
  (-)^{m\cdot n_\bo{Q}}
  \la(Q_1)\etc\la(Q_k)
\\
&
  \la(q_{i_1}\cd q_{i_n})
=
  \sum_{k=1}^n
  (-)^{k+1}
  (k-1)!
  \sum_{(Q_1\etc Q_k)}^{\mc{P}_k(q_{i_1}\etc q_{i_n})}
  (-)^{m\cdot n_\bo{Q}}
  \g(Q_1)\etc\g(Q_k)
\end{align}
where $m=0$ for commuting random variables and $m=1$ for particle-hole operators.}
\begin{addmargin}[1em]{0em}
Proof:
The generating functions are
\begin{align*}
&
  M(\bm\a)
=
  \expbox K(\bm\a)
=
  1
+
  \sum_n
  \sum_{i_1\cd i_n}
  \fr{\a_{i_1}\cd \a_{i_n}}{n!}
  \g(q_{i_1}\cd q_{i_n})
\\
&
  K(\bm\a)
=
  \logbox M(\bm\a)
=
  0
+
  \sum_n
  \sum_{i_1\cd i_n}
  \fr{\a_{i_1}\cd \a_{i_n}}{n!}
  \la(q_{i_1}\cd q_{i_n})
\end{align*}
where $\{\a_i\}$ are either ordinary numbers or Grassmann numbers.
Defining
$
  b_n
\equiv
  \sum_{i_1\cd i_n} \frac{\a_{i_1}\cd\a_{i_n}}{n!}\g(q_{i_1}\cd q_{i_n})
$
and
$
  c_n
\equiv
  \sum_{i_1\cd i_n} \frac{\a_{i_1}\cd\a_{i_n}}{n!}\la(q_{i_1}\cd q_{i_n})
$,
and Taylor expansion coefficients $T_k^{\expbox}=\frac{1}{k!}$ and $T_k^{\logbox}=\frac{(-)^{k+1}}{k}$, these can be expanded as
\begin{align*}
&
  M(\bm\a)
=
  \expbox(0+\sum_n c_n)
=
  1
+
  \sum_k
  T_k^{\expbox}
  \sum_{n_1\cd n_k}
  c_{n_1}\cd c_{n_k}
\\
&
  K(\bm\a)
=
  \logbox(1+\sum_n b_n)
=
  0
+
  \sum_k
  T_k^{\logbox}
  \sum_{n_1\cd n_k}
  b_{n_1}\cd b_{n_k}\ .
\end{align*}
To group terms in the summation in powers of the $\a_i$, the summations can be re-ordered using
\begin{align*}
  \sum_k
  \sum_{n_1\cd n_k}
  \a^{n_1+\cd+n_k}
  f_{n_1}\cd f_{n_k}
=
  \sum_n
  \sum_{k=1}^n
  \sum_{(n_1\cd n_k)}^{\mc{C}_k(n)}
  \a^n
  f_{n_1}\cd f_{n_k}
=
  \sum_n
  \sum_{k=1}^n
  k!
  \sum_{(n_1\cd n_k)}^{\mc{P}_k(n)}
  \a^n
  f_{n_1}\cd f_{n_k}
\end{align*}
where $\mc{C}_k(n)$ and $\mc{P}_k(n)$ are $k$-tuple integer compositions and partitions (respectively) of $n$.
That is, each $(n_1\cd n_k)$ is a tuple of positive integers less than $n$ such that $\sum_{i=1}^k n_i=n$.  $\mc{C}_k(n)$ counts different orderings separately.\footnote{Example: the \textit{partitions} of 3 are $\mc{P}(3)=\{(3),(2\ 1),(1\ 1\ 1)\}$ and its \textit{compositions} are $\mc{C}(3)=\{(3),(2\ 1),(1\ 2),(1\ 1\ 1)\}$.
The \textit{2-tuple partitions} of 3 are $\mc{P}_2(3)=\{(2\ 1)\}$ and its \textit{2-tuple compositions} are $\mc{C}_2(3)=\{(2\ 1),(1\ 2)\}$.}
This rearrangement gives
\begin{align*}
&
  M(\bm\a)
=
  1
+
  \sum_n
  \sum_{k=1}^n
  k! T_k^{\expbox}
  \sum_{(n_1\cd n_k)}^{\mc{P}_k(n)}
  c_{n_1}\cd c_{n_k}
\\
&
  K(\bm\a)
=
  0
+
  \sum_n
  \sum_{k=1}^n
  k! T_k^{\logbox}
  \sum_{(n_1\cd n_k)}^{\mc{P}_k(n)}
  b_{n_1}\cd b_{n_k}
\end{align*}
and expanding each $b_{n_i}$ and $c_{n_i}$ produces
\begin{align*}
&
  M(\bm\a)
=
  1
+
  \sum_n
  \sum_{i_1\cd i_n}
  \a_{i_1}\cd \a_{i_n}
  \sum_{k=1}^n
  k! T_k^{\expbox}
  \sum_{(n_1\cd n_k)}^{\mc{P}_k(n)}
  \fr{1}{n_1!\cd n_k!}
  \la(q_{i_1}\cd q_{i_{n_1}})\cd\la(q_{i_{n-n_k+1}}\cd q_{i_n})
\\
&
  K(\bm\a)
=
  0
+
  \sum_n
  \sum_{i_1\cd i_n}
  \a_{i_1}\cd \a_{i_n}
  \sum_{k=1}^n
  k! T_k^{\logbox}
  \sum_{(n_1\cd n_k)}^{\mc{P}_k(n)}
  \fr{1}{n_1!\cd n_k!}
  \g(q_{i_1}\cd q_{i_{n_1}})\cd\g(q_{i_{n-n_k+1}}\cd q_{i_n})\ .
\end{align*}
Taking derivatives with respect to the generator arguments, we find
\begin{align*}
&
  \left.
  \pd{^n M(\bm\a)}{\a_{i_n}\cd\pt\a_{i_1}}
  \right|_{\bm\a=0}
=
  \sum_{k=1}^n
  k! T_k^{\expbox}
  \sum_{(n_1\cd n_k)}^{\mc{P}_k(n)}
  \fr{1}{n_1!\cd n_k!}
  \sum_{\pi}^{\mr{S}_n}
  \e_\pi^m
  \la(q_{i_{\pi(1)}}\cd q_{i_{\pi(n_1)}})\cd\la(q_{i_{\pi(n-n_k+1)}}\cd q_{i_{\pi(n)}})
\\
&
  \left.
  \pd{^n K(\bm\a)}{\a_{i_n}\cd\pt\a_{i_1}}
  \right|_{\bm\a=0}
=
  \sum_{k=1}^n
  k! T_k^{\logbox}
  \sum_{(n_1\cd n_k)}^{\mc{P}_k(n)}
  \fr{1}{n_1!\cd n_k!}
  \sum_{\pi}^{\mr{S}_n}
  \e_\pi^m
  \g(q_{i_{\pi(1)}}\cd q_{i_{\pi(n_1)}})\cd\g(q_{i_{\pi(n-n_k+1)}}\cd q_{i_{\pi(n)}})
\end{align*}
where $m=0$ when the $\a_i$ are ordinary variables and $m=1$ when they are Grassmann variables.
The summations over integer partitions can be re-written in terms of product partitions of $q_{i_1}\cd q_{i_n}$, defined by analogy with set partitions.
\begin{align*}
  \sum_{(n_1\cd n_k)}^{\mc{P}_k(n)}
  \fr{1}{n_1!\cd n_k!}
  \sum_{\pi}^{\mr{S}_n}
  \e_\pi^m
  f(q_{i_{\pi(1)}}\cd q_{i_{\pi(n_1)}})\cd f(q_{i_{\pi(n-n_k+1)}}\cd q_{i_{\pi(n)}})
=
  \sum_{(Q_1\cd Q_k)}^{\mc{P}_k(q_{i_1}\cd q_{i_n})}
  (-)^{m\cdot n_\bo{Q}}
  f(Q_1)\cd f(Q_k)
\end{align*}
This comes from the fact that for each product $f(q_{i_{\pi(1)}}\cd q_{i_{\pi(n_1)}})\cd f(q_{i_{\pi(n-n_k+1)}}\cd q_{i_{\pi(n)}})$ there are exactly $n_1!\cd n_k!$ permutations of the operators within each argument, and these terms are equal (up to a phase factor).
These symmetries of $\g$ and $\la$ come from the fact that $\ds\pd{^n}{\a_1\cd \pt\a_n}=\e_\pi^m\pd{^n}{\a_{\pi(1)}\cd \pt\a_{\pi(n)}}$ for all $\pi\in\mr{S}_n$.
This rearrangement leaves
\begin{align*}
&
  \left.
  \pd{^n M(\bm\a)}{\a_{i_n}\cd\pt\a_{i_1}}
  \right|_{\bm\a=0}
=
  \sum_{k=1}^n
  k! T_k^{\expbox}
  \sum_{(Q_1\cd Q_k)}^{\mc{P}_k(q_{i_1}\cd q_{i_n})}
  (-)^{m\cdot n_\bo{Q}}
  \la(Q_1)\cd \la(Q_k)
\\
&
  \left.
  \pd{^n K(\bm\a)}{\a_{i_n}\cd\pt\a_{i_1}}
  \right|_{\bm\a=0}
=
  \sum_{k=1}^n
  k! T_k^{\logbox}
  \sum_{(Q_1\cd Q_k)}^{\mc{P}_k(q_{i_1}\cd q_{i_n})}
  (-)^{m\cdot n_\bo{Q}}
  \g(Q_1)\cd \g(Q_k)
\end{align*}
which, on plugging in $k!T_k^{\logbox}=(-)^{k+1}(k-1)!$ and $k!T_k^{\expbox}=1$, gives the final result:
\begin{align*}
&
  \g(q_{i_1}\cd q_{i_n})
\equiv
  \left.
  \pd{^n M(\bm\a)}{\a_{i_n}\cd\pt\a_{i_1}}
  \right|_{\bm\a=0}
=
  \sum_{k=1}^n
  \sum_{(Q_1\cd Q_k)}^{\mc{P}_k(q_{i_1}\cd q_{i_n})}
  (-)^{m\cdot n_\bo{Q}}
  \la(Q_1)\cd \la(Q_k)
\\
&
  \la(q_{i_1}\cd q_{i_n})
\equiv
  \left.
  \pd{^n K(\bm\a)}{\a_{i_n}\cd\pt\a_{i_1}}
  \right|_{\bm\a=0}
=
  \sum_{k=1}^n
  (-)^{k+1}(k-1)!
  \sum_{(Q_1\cd Q_k)}^{\mc{P}_k(q_{i_1}\cd q_{i_n})}
  (-)^{m\cdot n_\bo{Q}}
  \g(Q_1)\cd \g(Q_k)\ .
\end{align*}
\end{addmargin}
\end{pro}



\section{Wick's theorem}


\begin{thm}
\label{wick-thm}
\bmit{Wick's Theorem, $Q=\NO{Q}+\NO{\ol{Q}}$ (time-independent).}
\textit{Any string $Q$ of particle-hole operators is equal to $\NO{Q}+\NO{\ol{Q}}$, its normal-ordered form plus the sum of all possible contractions.}
\end{thm}

\begin{cor}
\label{wick-product}
\bmit{Wick's Theorem for operator products.}
\textit{
Given a pair $Q, Q'$ of particle-hole operator strings already in normal order, the product of their normal orderings is given by $\NO{Q}\NO{Q'}=\NO{QQ'}+\NO{\ctr{}{Q}{}{Q}{'}QQ'}$.}
\end{cor}

\begin{cor}
\label{wick-vacuum-expectation}
\bmit{$\ip{\vac|Q|\vac}=\NO{\ol{\ol{Q}}}$.}
\textit{The vacuum expectation value of a string of particle-hole operators equals the sum of its complete contractions.}
\end{cor}


\end{document}
