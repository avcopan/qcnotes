\documentclass[11pt,fleqn]{article}
\usepackage[cm]{fullpage}
\usepackage{mathtools} %includes amsmath
\usepackage{amsfonts}
\usepackage{bm}
\usepackage{xfrac}
\usepackage{url}
%greek letters
\renewcommand{\a}{\alpha}    %alpha
\renewcommand{\b}{\beta}     %beta
\newcommand{\g}{\gamma}      %gamma
\newcommand{\G}{\Gamma}      %Gamma
\renewcommand{\d}{\delta}    %delta
\newcommand{\D}{\Delta}      %Delta
\newcommand{\e}{\varepsilon} %epsilon
\newcommand{\ev}{\epsilon}   %epsilon*
\newcommand{\z}{\zeta}       %zeta
\newcommand{\h}{\eta}        %eta
\renewcommand{\th}{\theta}   %theta
\newcommand{\Th}{\Theta}     %Theta
\newcommand{\io}{\iota}      %iota
\renewcommand{\k}{\kappa}    %kappa
\newcommand{\la}{\lambda}    %lambda
\newcommand{\La}{\Lambda}    %Lambda
\newcommand{\m}{\mu}         %mu
\newcommand{\n}{\nu}         %nu %xi %Xi %pi %Pi
\newcommand{\p}{\rho}        %rho
\newcommand{\si}{\sigma}     %sigma
\newcommand{\siv}{\varsigma} %sigma*
\newcommand{\Si}{\Sigma}     %Sigma
\renewcommand{\t}{\tau}      %tau
\newcommand{\up}{\upsilon}   %upsilon
\newcommand{\f}{\phi}        %phi
\newcommand{\F}{\Phi}        %Phi
\newcommand{\x}{\chi}        %chi
\newcommand{\y}{\psi}        %psi
\newcommand{\Y}{\Psi}        %Psi
\newcommand{\w}{\omega}      %omega
\newcommand{\W}{\Omega}      %Omega
%ornaments
\newcommand{\eth}{\ensuremath{^\text{th}}}
\newcommand{\rst}{\ensuremath{^\text{st}}}
\newcommand{\ond}{\ensuremath{^\text{nd}}}
\newcommand{\ord}[1]{\ensuremath{^{(#1)}}}
\newcommand{\dg}{\ensuremath{^\dagger}}
\newcommand{\bigo}{\ensuremath{\mathcal{O}}}
\newcommand{\tl}{\ensuremath{\tilde}}
\newcommand{\ol}[1]{\ensuremath{\overline{#1}}}
\newcommand{\ul}[1]{\ensuremath{\underline{#1}}}
\newcommand{\op}[1]{\ensuremath{\hat{#1}}}
\newcommand{\ot}{\ensuremath{\otimes}}
\newcommand{\wg}{\ensuremath{\wedge}}
%text
\newcommand{\tr}{\ensuremath{\hspace{1pt}\mathrm{tr}\hspace{1pt}}}
\newcommand{\Alt}{\ensuremath{\mathrm{Alt}}}
\newcommand{\sgn}{\ensuremath{\mathrm{sgn}}}
\newcommand{\occ}{\ensuremath{\mathrm{occ}}}
\newcommand{\vir}{\ensuremath{\mathrm{vir}}}
\newcommand{\spn}{\ensuremath{\mathrm{span}}}
\newcommand{\vac}{\ensuremath{\mathrm{vac}}}
\newcommand{\bs}{\ensuremath{\text{\textbackslash}}}
\newcommand{\im}{\ensuremath{\mathrm{im}\hspace{1pt}}}
\renewcommand{\sp}{\hspace{30pt}}
%dots
\newcommand{\ld}{\ensuremath{\ldots}}
\newcommand{\cd}{\ensuremath{\cdots}}
\newcommand{\vd}{\ensuremath{\vdots}}
\newcommand{\dd}{\ensuremath{\ddots}}
\newcommand{\etc}{\ensuremath{\mathinner{\mkern-1mu\cdotp\mkern-2mu\cdotp\mkern-2mu\cdotp\mkern-1mu}}}
%fonts
\newcommand{\bmit}[1]{{\bfseries\itshape\mathversion{bold}#1}}
\newcommand{\mc}[1]{\ensuremath{\mathcal{#1}}}
\newcommand{\mb}[1]{\ensuremath{\mathbb{#1}}}
\newcommand{\mf}[1]{\ensuremath{\mathfrak{#1}}}
\newcommand{\mr}[1]{\ensuremath{\mathrm{#1}}}
\newcommand{\bo}[1]{\ensuremath{\mathbf{#1}}}
%styles
\newcommand{\ts}{\textstyle}
\newcommand{\ds}{\displaystyle}
\newcommand{\phsub}{\ensuremath{_{\phantom{p}}}}
\newcommand{\phsup}{\ensuremath{^{\phantom{p}}}}
%fractions, derivatives, parentheses, brackets, etc.
\newcommand{\pr}[1]{\ensuremath{\left(#1\right)}}
\newcommand{\brk}[1]{\ensuremath{\left[#1\right]}}
\newcommand{\fr}[2]{\ensuremath{\dfrac{#1}{#2}}}
\newcommand{\pd}[2]{\ensuremath{\frac{\partial#1}{\partial#2}}}
\newcommand{\pt}{\ensuremath{\partial}}
\newcommand{\br}[1]{\ensuremath{\langle#1|}}
\newcommand{\kt}[1]{\ensuremath{|#1\rangle}}
\newcommand{\ip}[1]{\ensuremath{\langle#1\rangle}}
\newcommand{\NO}[1]{\ensuremath{{\bm{:}}#1{}{\bm{:}}}}
\newcommand{\floor}[1]{\ensuremath{\left\lfloor#1\right\rfloor}}
\newcommand{\ceil}[1]{\ensuremath{\left\lceil#1\right\rceil}}
\usepackage{stackengine}
\newcommand{\GNO}[1]{\setstackgap{S}{0.7pt}\ensuremath{\Shortstack{\textbf{.} \textbf{.} \textbf{.}}#1\Shortstack{\textbf{.} \textbf{.} \textbf{.}}}}
\newcommand{\cmtr}[2]{\ensuremath{[\cdot,#2]^{#1}}}
\newcommand{\cmtl}[2]{\ensuremath{[#2,\cdot]^{#1}}}
%structures
\newcommand{\eqn}[1]{(\ref{#1})}
\newcommand{\ma}[1]{\ensuremath{\begin{bmatrix}#1\end{bmatrix}}}
\newcommand{\ar}[1]{\ensuremath{\begin{matrix}#1\end{matrix}}}
\newcommand{\miniar}[1]{\ensuremath{\begin{smallmatrix}#1\end{smallmatrix}}}
%contractions
\usepackage{simplewick}
\usepackage[nomessages]{fp}
\newcommand{\ctr}[6][0]{\FPeval\height{0.6+#1*0.5}\ensuremath{\contraction[\height ex]{#2}{#3}{#4}{#5}}}
\newcommand{\ccr}[4]{\ctr[0.7]{#1}{#2}{#3}{#4}{}\ctr{#1}{#2}{#3}{#4}}
\usepackage{calc}
\makeatletter
\def\@hspace#1{\begingroup\setlength\dimen@{#1}\hskip\dimen@\endgroup}
\makeatother
\newcommand{\halfphantom}[1]{\hspace{\widthof{#1}*\real{0.5}}}
\newcommand{\fullctr}[1]{\ensuremath{\contraction[0.5ex]{}{\vphantom{#1}}{\hphantom{#1}}{}{}{}\contraction[0.5ex]{}{\vphantom{#1}}{\halfphantom{#1}}{}{}{}#1}}
%math sections
\usepackage{cleveref}
\usepackage{amsthm}
\usepackage{thmtools}
\declaretheoremstyle[spaceabove=10pt,spacebelow=10pt,bodyfont=\small]{mystyle}
\theoremstyle{mystyle}
\newtheorem{dfn}{Definition}[section]
\crefname{dfn}{definition}{definitions}
\Crefname{dfn}{Def}{Defs}
\newtheorem{thm}{Theorem}[section]
\crefname{thm}{theorem}{theorems}
\Crefname{thm}{Thm}{Thms}
\newtheorem{cor}{Corollary}[section]
\crefname{cor}{corollary}{corollaries}
\Crefname{cor}{Cor}{Cors}
\newtheorem{lem}{Lemma}[section]
\crefname{lem}{lemma}{lemmas}
\Crefname{lem}{Lem}{Lems}
\newtheorem{rmk}{Remark}[section]
\crefname{rmk}{remark}{remarks}
\Crefname{rmk}{Rmk}{Rmks}
\newtheorem{pro}{Proposition}[section]
\crefname{pro}{proposition}{propositions}
\Crefname{pro}{Prop}{Props}
\newtheorem{ntt}{Notation}[section]
\crefname{ntt}{notation}{notations}
\Crefname{ntt}{Notation}{Notations}
\newcommand{\logbox}{\ensuremath{\text{\makebox[\widthof{exp}][c]{ln}}}}
\newcommand{\expbox}{\ensuremath{\text{\makebox[\widthof{exp}][c]{exp}}}}

\numberwithin{equation}{section}
\usepackage{cancel}
\usepackage{scrextend}

\newcommand{\hole}{\circ}
\newcommand{\ptcl}{\bullet}
\newcommand{\circled}[1]{\raisebox{.5pt}{\textcircled{\raisebox{-.9pt}{#1}}}}



%%%DOCUMENT%%%
\begin{document}


\begin{pro}
\label{excitation-operators-phase-rule}
\bmit{Sign rule for completely contracted excitation operators.}
\textit{Let $c$ be the number of intersections when straight lines are drawn between contracted pairs of upper and lower indices.
The phase factor of a completely contracted product of excitation operators is then $(-)^{c+h}$ where $h$ is the number of $\h$-contractions.}
\begin{addmargin}[1em]{0em}
Proof:
First, consider a completely contracted product of the form $\GNO{a_{q_1^{\hole1}\cd q_h^{\hole h}}^{p_1^{\hole1}\cd p_h^{\hole h}}a_{s_1^{\ptcl1}\cd s_k^{\ptcl k}}^{r_1^{\ptcl1}\cd r_k^{\ptcl k}}}$, where the imagined contraction lines are all parallel and therefore non-intersecting.
From \Cref{excitation-operator-phase}, we have that this is equal to $\GNO{a_{q_1^{\hole1}}^{p_1^{\hole1}}\cd a_{q_h^{\hole h}}^{p_h^{\hole h}}a_{s_1^{\ptcl1}}^{r_1^{\ptcl1}}a_{s_k^{\ptcl k}}^{r_k^{\ptcl k}}}=(-\h_{q_1}^{p_1})\cd(-\h_{q_h}^{p_h})\g_{s_1}^{r_1}\cd\g_{s_k}^{r_k}=(-)^h\h_{q_1}^{p_1}\cd\h_{q_h}^{p_h}\g_{s_1}^{r_1}\cd\g_{s_k}^{r_k}$.
All other completely contracted products of excitation operators are related to this one by permutations of the upper and lower indices.
Since each transposition of adjacent upper or lower indices changes the number of line crossings $c$ by exactly $\pm1$, the general case has an additional factor of $(-)^c$.
\end{addmargin}
\end{pro}

\begin{dfn}
\bmit{Open and closed loops.}
Consider the general case of an incomplete set of contractions in an excitation operator $\GNO{a_{q_1\cd q_m}^{p_1\cd p_m}}=\GNO{a_{q_1}^{p_1}\cd a_{q_m}^{p_m}}$.
Suppose at least one of the indices in $a_{q_i}^{p_i}$ is involved in a contraction and consider the other single excitation operators $a_{q_j}^{p_j}$, $a_{q_k}^{p_k}$, $a_{q_l}^{p_l}$, etc. connected to it via contractions.
This set of single-excitation operators forms either an \textit{open loop} such as $a_{q_i}^{p_i^\ptcl}a_{q_j^\ptcl}^{p_j}$ or $a_{q_i}^{p_i^\ptcl}a_{q_j^{\hole}}^{p_j}a_{q_k^\ptcl}^{p_k^{\hole}}$ or $a_{q_i^\hole}^{p_i^\ptcl}a_{q_j^\ptcl}^{p_j^{\ptcl\ptcl}}a_{q_k^{\ptcl\ptcl}}^{p_k}a_{q_l}^{p_l^\hole}$ where there is one lower uncontracted index and one upper uncontracted index, or it forms a \textit{closed loop} such as $a_{q_i^\ptcl}^{p_i^\ptcl}$ or $a_{q_i^\hole}^{p_i^\ptcl}a_{q_j^\ptcl}^{p_j^\hole}$ or $a_{q_i^\hole}^{p_i^\ptcl}a_{q_j^\ptcl}^{p_j^{\ptcl\ptcl}}a_{q_k^{\ptcl\ptcl}}^{p_k^{\hole\hole}}a_{q_l^{\hole\hole}}^{p_l^\hole}$ where there are no uncontracted indices.
\end{dfn}

\begin{rmk}
\bmit{General procedure for evaluating contracted excitation operators.}
Simple normal-ordered products of excitation operators can be evaluated by inspection.
For more complicated cases, an operator $\GNO{a_{q_1\cd q_m}^{p_1\cd p_m}}=\GNO{a_{q_1}^{p_1}\cd a_{q_m}^{p_m}}$ with contractions can be evaluated by the following systematic procedure:\\
\begin{addmargin}[1em]{0em}
\begin{tabular}{ll}
1. Group open and closed loops together by moving single-excitation operators $a_{q_i}^{p_i}$ through the string.
& $(+)$ \\
2. Close each open loop by a transposition that pairs its uncontracted operators, i.e. $a_{q_i}^{p_i^\ptcl}a_{q_j^\ptcl}^{p_j}=-a_{q_j^\ptcl}^{p_i^\ptcl}a_{q_i}^{p_j}$.
& $(-)$ \\
3. Evaluate each closed loop using \Cref{excitation-operators-phase-rule}.
& $(-)^{c+h}$ 
\end{tabular}
\end{addmargin}\ \\[5pt]
where the signs on the right are the phase factors for the required permutations.
\end{rmk}


\begin{ntt}
\label{compound-indices}
\bmit{Compound indices.}
Capital letters will here be used to denote compound indices, so that $P$ stands for $p_1\cd p_m$, $a_Q^P$ stands for $a_{q_1\cd q_m}^{p_1\cd p_m}$, and $\tl{a}_Q^P$ stands for $\tl{a}_{q_1\cd q_m}^{p_1\cd p_m}$.
Using this notation, the algebraic rearrangements of \Cref{excitation-operator-phase} can be generalized to 
$
  \NO{a_{Q_1}^{P_1}\cd a_{Q_N}^{P_N}}
=
  a_{Q_1\cd Q_N}^{P_1\cd P_N}
$
and
$
  \GNO{\tl{a}_{Q_1}^{P_1}\cd\tl{a}_{Q_N}^{P_N}}
=
  \tl{a}_{Q_1\cd Q_N}^{P_1\cd P_N}
$.
Furthermore, contraction symbols $\hole$ and $\ptcl$ on these compound indices will represent sums over unique sets of $\h$- and $\g$-contractions, respectively.
For example, the $\F$-normal Wick expansion of $a_Q^P$ can be written as $\GNO{a_Q^P}+\GNO{\ol{a_Q^P}}=\tl{a}_Q^P+\GNO{a_{Q^\ptcl}^{P^\ptcl}}$.
Similarly, the product expansion for $\tl{a}_Q^P\tl{a}_S^R$ can be written as $\GNO{a_Q^Pa_S^R}+\GNO{\ctr[0.7]{}{a}{_Q^P}{a}{_S^R}a_Q^Pa_S^R}=\tl{a}_{QS}^{PR}+\GNO{a_{Q^\hole}^{P^\ptcl}a_{S^\ptcl}^{R^\hole}}$.
\end{ntt}


\begin{pro}
\label{aPQ-contractions-closed-form}
\bmit{Formula for $a_Q^P$ Wick expansion.}
\textit{
The Wick expansion for $a_Q^P$ is given by
\begin{align*}
&&
  \tl{a}_Q^P
+
  \GNO{a_{Q^\ptcl}^{P^\ptcl}}
=
  \sum_{g=0}^{m}
  \op{P}_{(q_1/\cd/q_g/q_{g+1}\cd q_m)}
        ^{(p_1\cd p_g/p_{g+1}\cd p_m)}
  \g_{q_1}^{p_1}\cd\g_{q_g}^{p_g}
  \tl{a}_{q_{g+1}\cd q_m}^{p_{g+1}\cd p_m}
\end{align*}
using \Cref{index-permutation-operator}.}
\begin{addmargin}[1em]{0em}
Proof:
Distributing contraction labels $\ptcl1,\cd,\ptcl g$ in all possible ways on the upper and lower indices of $a_Q^P$
is equivalent to
$
  \op{P}_{(q_1/\cd/q_g/q_{g+1}\cd q_m)}
        ^{(p_1/\cd/p_g/p_{g+1}\cd p_m)}
  \GNO{a_{q_1^{\ptcl1}\cd q_g^{\ptcl g}q_{g+1}\cd q_m}
        ^{p_1^{\ptcl1}\cd p_g^{\ptcl g}p_{g+1}\cd p_m}}
$
which equals
$
  \op{P}_{(q_1/\cd/q_g/q_{g+1}\cd q_m)}
        ^{(p_1/\cd/p_g/p_{g+1}\cd p_m)}
  \g_{q_1}^{p_1}\cd\g_{q_g}^{p_g}
  \tl{a}_{q_{g+1}\cd q_m}^{p_{g+1}\cd p_m}
$.
However, each term in this expansion is one of $g!$ equivalent terms related by relabeling of the contractions $\ptcl1,\cd,\ptcl g\mapsto\ptcl\pi(1),\cd,\ptcl\pi(g)$.
These redundant contractions can be removed by omitting the permutations between either the upper contracted indices or the lower contracted indices.
Choosing the first option gives
$
  \op{P}_{(q_1/\cd/q_g/q_{g+1}\cd q_m)}
        ^{(p_1\cd p_g/p_{g+1}\cd p_m)}
  \g_{q_1}^{p_1}\cd\g_{q_g}^{p_g}
  \tl{a}_{q_{g+1}\cd q_m}^{p_{g+1}\cd p_m}
$
as the unique $g$-tuple $\g$-contractions of $a_Q^P$.
Summing over $g$ from $0$ to $m$ gives
$
  \tl{a}_Q^P
+
  \GNO{a_{Q^\ptcl}^{P^\ptcl}}
$.
\end{addmargin}
\end{pro}


\begin{ntt}
\label{extended-permutation-ops}
\bmit{More index permutation operators.}
Let arguments of $\op{P}$ separated by bars denote independent sets of permutations, i.e. $\op{P}^{(P_1/\cd/P_L|R_1/\cd/R_N)}_{(Q_1/\cd/Q_M|S_1/\cd/S_O)}=\op{P}^{(P_1/\cd/P_L)}_{(Q_1/\cd/Q_M)}\op{P}^{(R_1/\cd/R_N)}_{(S_1/\cd/S_O)}$.
\end{ntt}

\begin{ntt}
\label{compound-index-subsets}
\bmit{Compound index subsets.}
For a compound index $P$ representing $p_1\cd p_m$, let $P_j^k$ denote $p_{j+1}\cd p_k$ when $j<k$.
When $j=k$, let $P_j^k$ be an empty placeholder.
Furthermore, let $P^k$ denote $p_1\cd p_k$ and let $P_j$ denote $p_j\cd p_m$.
For example, the statement of \Cref{aPQ-contractions-closed-form} using this notation would be
$
  \GNO{a_{Q^\ptcl}^{P^\ptcl}}
=
  \sum_{g=1}^{m}
  \op{P}_{(q_1/\cd/q_g/Q_g)}
        ^{(P^g/P_g)}
  \g_{q_1}^{p_1}\cd\g_{q_g}^{p_g}
  \tl{a}_{Q_g}^{P_g}
$.
\end{ntt}

\begin{pro}
\label{aPQaSR-wick-expansion}
\bmit{Formula for for $\tl{a}_Q^P\tl{a}_S^R$ Wick expansion.}
\textit{
The Wick expansion for $\tl{a}_Q^P\tl{a}_S^R$ is given by
\begin{align*}
&&
  \tl{a}_{QS}^{PR}
+
  \GNO{a_{Q^\hole}^{P^\ptcl}a_{S^\ptcl}^{R^\hole}}
=&\
  \sum_{g\leq h}^{\min(m,n)}
  \op{P}_{(Q^h/Q_h|s_1/\cd/s_g/S_g)}
        ^{(P^g/P_g|r_1/\cd/r_h/R_h)}
  \g_{s_1}^{p_1}\cd\g_{s_g}^{p_g}
  \h_{q_1}^{r_1}\cd\h_{q_h}^{r_h}
  \tl{a}_{S_g^hQ_hS_h}
        ^{P_g^hP_hR_h}
\\&&+&\
  \sum_{g>h}^{\min(m,n)}
  (-)^{g+h}
  \op{P}_{(Q^h/Q_h|s_1/\cd/s_g/S_g)}
        ^{(P^g/P_g|r_1/\cd/r_h/R_h)}
  \g_{s_1}^{p_1}\cd\g_{s_g}^{p_g}
  \h_{q_1}^{r_1}\cd\h_{q_h}^{r_h}
  \tl{a}_{Q_h^gQ_gS_g}
        ^{R_h^gP_gR_g}
\end{align*}
using \Cref{extended-permutation-ops} and \Cref{compound-index-subsets}.
}
\begin{addmargin}[1em]{0em}
Proof:
By the same reasoning as was used in the proof for of \Cref{aPQ-contractions-closed-form}, the unique sets of simultaneous ($g$,$h$)-tuple ($\g$,$\h$)-contractions of $\tl{a}_Q^P\tl{a}_s^R$ are given by
$
  \op{P}_{(s_1/\cd/s_g/S_g)}
        ^{(P^g/P_g)}
  \op{P}_{(Q^h/Q_h)}
        ^{(r_1/\cd/r_h/R_h)}
  \GNO{
    a_{q_1^{\hole1}\cd q_h^{\hole h}Q_h}
     ^{p_1^{\ptcl1}\cd p_g^{\ptcl g}P_g}
    a_{s_1^{\ptcl1}\cd s_g^{\ptcl g}S_g}
     ^{r_1^{\hole1}\cd r_h^{\hole h}R_h}
  }
$
and the permutation operators can be combined into one as
$
  \op{P}_{(Q^h/Q_h|s_1/\cd/s_g/S_g)}
        ^{(P^g/P_g|r_1/\cd/r_h/R_h)}
$.
Summing over $g$ and $h$ from $0$ to $\min(m,n)$ gives
$
  \tl{a}_{QS}^{PR}
+
  \GNO{a_{Q^\hole}^{P^\ptcl}a_{S^\ptcl}^{R^\hole}}
$.
The term
$
  \GNO{
    a_{q_1^{\hole1}\cd q_h^{\hole h}Q_h}
     ^{p_1^{\ptcl1}\cd p_g^{\ptcl g}P_g}
    a_{s_1^{\ptcl1}\cd s_g^{\ptcl g}S_g}
     ^{r_1^{\hole1}\cd r_h^{\hole h}R_h}
  }
$
can be evaluated by transposing $q_i\leftrightarrow s_i$ along with their contraction labels for $1\leq i\leq\max(g,h)$ in order to pair the contractions vertically, which gives
\begin{align*}
&&&
  (-)^h
  \g_{s_1}^{p_1}\cd\g_{s_g}^{p_g}
  (-\h_{q_1}^{r_1})\cd(-\h_{q_h}^{r_h})
  \tl{a}_{S_g^hQ_hS_h}
        ^{P_g^hP_hR_h}
\sp\text{when $g<h$}
\\
&&
\text{or}\sp
&
  (-)^g
  \g_{s_1}^{p_1}\cd\g_{s_g}^{p_g}
  (-\h_{q_1}^{r_1})\cd(-\h_{q_h}^{r_h})
  \tl{a}_{Q_h^gQ_gS_g}
        ^{R_h^gP_gR_g}
\sp\text{\ when $g>h$}
\end{align*}
noting that $a_{q^\hole}^{p^\hole}=-\h_q^p$.
\end{addmargin}
\end{pro}

\begin{rmk}
Using \Cref{aPQaSR-wick-expansion}, the $\F$-normal Wick expansions for products of $\tl{a}_q^p$ and $\tl{a}_{rs}^{pq}$ are as follows.
{\setlength{\mathindent}{1em}
\begin{align*}
  \tl{a}_q^p
  \tl{a}_s^r
=&\
  \tl{a}_{qs}^{pr}
+
  \h_q^r\tl{a}_s^p
+
  \g_s^p\h_q^r
-
  \g_s^p\tl{a}_q^r
\\
  \tl{a}_{q_1}^{p_1}
  \tl{a}_{s_1s_2}^{r_1r_2}
=&\
  \tl{a}_{q_1s_1s_2}^{p_1r_1r_2}
+
  \op{P}^{(r_1/r_2)}
  \h_{q_1}^{r_1}
  \tl{a}_{s_1s_2}^{p_1r_1}
+
  \op{P}^{(r_1/r_2)}_{(s_1/s_2)}
  \g_{s_1}^{p_1}
  \h_{q_1}^{r_1}
  \tl{a}_{s_2}^{r_2}
-
  \op{P}_{(s_1/s_2)}
  \g_{s_1}^{p_1}
  \tl{a}_{q_1s_2}^{r_1r_2}
\\
  \tl{a}_{q_1q_2}^{p_1p_2}
  \tl{a}_{s_1s_2}^{r_1r_2}
=&\
  \tl{a}_{q_1q_2s_1s_2}^{p_1p_2r_1r_2}
+
  \op{P}^{(r_1/r_2)}_{(q_1/q_2)}
  \h_{q_1}^{r_1}
  \tl{a}_{s_1q_2s_2}^{p_1p_2r_2}
+
  \op{P}^{(r_1/r_2)}
  \h_{q_1}^{r_1}\h_{q_2}^{r_2}
  \tl{a}_{s_1s_2}^{p_1p_2}
+
  \op{P}^{(p_1/p_2|r_1/r_2)}_{(q_1/q_2|s_1/s_2)}
  \g_{s_1}^{p_1}\h_{q_1}^{r_1}
  \tl{a}_{q_2s_2}^{p_2r_2}
+
  \op{P}^{(p_1/p_2|r_1/r_2)}_{(s_1/s_2)}
  \g_{s_1}^{p_1}\h_{q_1}^{r_1}\h_{q_2}^{r_2}
  \tl{a}_{s_2}^{p_2}
\\
+&\
  \op{P}^{(r_1/r_2)}_{(s_1/s_2)}
  \g_{s_1}^{p_1}\g_{s_2}^{p_2}\h_{q_1}^{r_1}\h_{q_2}^{r_2}
-
  \op{P}^{(p_1/p_2)}_{(s_1/s_2)}
  \g_{s_1}^{p_1}
  \tl{a}_{q_1q_2s_2}^{r_1p_2r_2}
+
  \op{P}_{(s_1/s_2)}
  \g_{s_1}^{p_1}\g_{s_2}^{p_2}
  \tl{a}_{q_1q_2}^{r_1r_2}
+
  \op{P}^{(r_1/r_2)}_{(q_1/q_2|s_1/s_2)}
  \g_{s_1}^{p_1}\g_{s_2}^{p_2}
  \h_{q_1}^{r_1}
  \tl{a}_{q_2}^{r_2}
\end{align*}}
Products expansions of higher excitation operators become quite lengthy, even with the use of index permutation operators.
\end{rmk}

\end{document}

