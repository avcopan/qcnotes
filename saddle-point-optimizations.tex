\documentclass[11pt]{article}
\usepackage[cm]{fullpage}
\usepackage{amsmath}

\begin{document}

\paragraph{Newton-Raphson step.}
Quadratic expansion
\[
    E_*
    \approx
    E
    +
    \Delta\mathbf{x}_*^\mathrm{t}
    \mathbf{g}
    +
    \tfrac{1}{2}
    \Delta\mathbf{x}_*^\mathrm{t}
    \mathbf{H}
    \Delta\mathbf{x}_*
    \qquad
    \mathbf{x}_*
    \equiv
    \mathbf{x}
    +
    \Delta\mathbf{x}_*
\]
Linear gradient expansion
\[
    \mathbf{g}_*
    =
    \mathbf{g}
    +
    \mathbf{H}
    \Delta\mathbf{x}_*
\]
Setting the gradient at the next point to zero gives the Newton-Raphson step
\[
    \mathbf{g}_*
    \overset{!}{=}
    \mathbf{0}
    \implies
    \Delta\mathbf{x}_*
    =
    -
    \mathbf{H}^{-1}
    \mathbf{g}
\]
For a perfectly quadratic surface, the Hessian is constant and the
Newton-Raphson step takes us directly to the stationary point.
On a surface with cubic and higher-order terms, this step can be repeated
iteratively until we are close enough to the stationary point that the region
separating us from it is approximately quadratic.


\paragraph{Quasi-Newton condition.}
For points
\(
    \mathbf{x}
\)
and
\(
    \mathbf{x}_0
\)
sharing a locally quadratic region with each other, the change in the gradient
between them is described by the following.
\[
    \mathbf{H}
    \Delta\mathbf{x}
    \approx
    \mathbf{H}_0
    \Delta\mathbf{x}
    \approx
    \Delta\mathbf{g}
    \qquad
    \begin{array}{r@{\,}l}
        \Delta\mathbf{x}
        &\equiv
        \mathbf{x} - \mathbf{x}_0
        \\
        \Delta\mathbf{g}
        &\equiv
        \mathbf{g} - \mathbf{g}_0
    \end{array}
\]
This can be used to determine an approximation
\(
    \tilde{\mathbf{H}}
    \approx
    \mathbf{H}
\)
to the Hessian at
\(
    \mathbf{x}
\)
using the Hessian at the other point.
Namely, we require the approximation to satisfy
\[
    \tilde{\mathbf{H}}
    \Delta\mathbf{x}
    \overset{!}{=}
    \Delta\mathbf{g}
    \qquad
    \tilde{\mathbf{H}}
    =
    \mathbf{H}_0
    +
    \Delta\tilde{\mathbf{H}}
\]
which is known as the {\itshape quasi-Newton condition}.
If the dimension is \(d\), then we have \(d\) linear equations and \(d^2\)
elements in the correction matrix, so this equation is underdetermined.
One simple choice is a rank-1 approximation 
\[
    \Delta\tilde{\mathbf{H}}
    =
    \eta\,
    \mathbf{e}
    \mathbf{e}^\mathrm{t}
\]

\end{document}
