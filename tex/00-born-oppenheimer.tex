\documentclass[11pt]{article}
\usepackage[cm]{fullpage}
%%AVC PACKAGES
\usepackage{avcgreek}
\usepackage{avcfonts}
\usepackage{avcmath}
\usepackage[skip=9pt plus 2pt minus 7pt]{avcthm}
\usepackage{qcmacros}
\usepackage{goldstone}

\begin{document}

\setlength{\abovedisplayskip}{5pt}
\setlength{\belowdisplayskip}{5pt}


\section*{The molecular Schr\"odinger equation}

\begin{rmk}
The non-relativistic Hamiltonian of a system of nuclei and electrons includes a kinetic energy operator for each particle and a Coulomb operator for each particle pair.\footnote{
  For $\mr{x}=\mr{y}$ one must restrict the sum on the right to unique $i,j$ pairs in order to avoid double-counting.}
\begin{align}
  \op{H}(\bo{r}_\mr{e},\bo{r}_\mr{n})
=
  \op{T}^\mr{e}
+
  \op{T}^\mr{n}
+
  V^\mr{e}(\bo{r}_\mr{e})
+
  V^\mr{n}(\bo{r}_\mr{n})
+
  V^{\mr{e},\mr{n}}(\bo{r}_\mr{e},\bo{r}_\mr{n})
&&
  \op{T}^\mr{x}
=
  \fr{1}{2m_\mr{x}}
  \sum
  \op{\bo{p}}_{\mr{x},i}^2
&&
  V^{\mr{x},\mr{y}}(\bo{r}_\mr{x},\bo{r}_\mr{y})
=
  \fr{1}{4\pi \e_0}
  \sum
  \fr{
    q_\mr{x}q_\mr{y}
  }{
    \|\bo{r}_{\mr{x},i} - \bo{r}_{\mr{y},j}\|^2
  }
\end{align}
The state of this system is described by a wavefunction,
$\Y_\a^\mr{tot}(\bo{r}_\mr{e},\bo{r}_\mr{n}, t)$, which is governed by the Schr\"odinger equation.
\begin{align}
  \op{H}(\bo{r}_\mr{e},\bo{r}_\mr{n})
  \Y_\a^\mr{tot}(\bo{r}_\mr{e},\bo{r}_\mr{n}, t)
=
  i
  \hbar
  \pd{
    \Y_\a^\mr{tot}(\bo{r}_\mr{e},\bo{r}_\mr{n}, t)
  }{
    t
  }
\end{align}
In the absence of external fields the system exists in a \textit{stationary state}.
Stationary states are characterized by a static wavefunction $\Y_\a^\mr{tot}(\bo{r}_\mr{e},\bo{r}_\mr{n})$ and a conserved energy expectation value $E_\a^\mr{tot}$ that controls its frequency of oscillation in time.
These quantities are governed by a Hamiltonian eigenvalue equation, which is the \textit{time-independent Schr\"odinger equation}.
\begin{align}
\label{eq:molecular-schrodinger-equation}
  \op{H}(\bo{r}_\mr{e},\bo{r}_\mr{n})
  \Y_\a^\mr{tot}(\bo{r}_\mr{e},\bo{r}_\mr{n})
=
  E_\a^\mr{tot}
  \Y_\a^\mr{tot}(\bo{r}_\mr{e},\bo{r}_\mr{n})
&&
  \Y_\a^\mr{tot}(\bo{r}_\mr{e},\bo{r}_\mr{n}, t)
=
  \Y_\a^\mr{tot}(\bo{r}_\mr{e},\bo{r}_\mr{n})
  e^{-\fr{i}{\hbar}E_\a^\mr{tot}\,t}
\end{align}
If $\mc{H}^\mr{e}\equiv\{\F_k(\bo{r}_\mr{e})\,|\,k\}$ is a complete basis of functions in $\bo{r}_\mr{e}$, the time-independent wavefunction can be expanded as follows.
\begin{align}
\label{eq:general-wavefunction-expansion}
  \Y_\a^\mr{tot}(\bo{r}_\mr{e}, \bo{r}_\mr{n})
=
\ts{
  \sum_{k'}
  \F_{k'}(\bo{r}_\mr{e})\,
  \x_{k',\a}(\bo{r}_\mr{n})
}
\end{align}
Substituting this into equation~\ref{eq:molecular-schrodinger-equation} and integrating with $\F_k^*(\bo{r}_\mr{e})$ yields an equation for the unknown nuclear functions.
\begin{align}
\label{eq:general-nuclear-schrodinger-equation}
  \op{\bo{H}}(\bo{r}_\mr{n})
  \bm{\x}_\a(\bo{r}_\mr{n})
=
  E_\a^\mr{tot}
  \bm{\x}_\a(\bo{r}_\mr{n})
&&
  \op{H}_{k,k'}(\bo{r}_\mr{n})
\equiv
  \int
  d^{n_\mr{e}}\bo{r}_\mr{e}\,\,
  \F_k^*(\bo{r}_\mr{e})
  \op{H}(\bo{r}_\mr{e},\bo{r}_\mr{n})
  \F_{k'}(\bo{r}_\mr{e})
&&
  \bm{\x}_\a(\bo{r}_\mr{n})
=
\ma{
  \x_{0,\a}(\bo{r}_\mr{n})\\
  \x_{1,\a}(\bo{r}_\mr{n})\\
  \vdots
}
\end{align}
This operator matrix equation can be solved by further expanding each entry in a basis for nuclear function space, yielding a supermatrix eigenvalue equation with scalar coefficients.
Exact solution of this equation is computationally intractable because the electronic and nuclear function spaces are infinite-dimensional.
In practice this can be resolved by ``coarse-graining'' each space with a finite subset of carefully chosen functions.
\end{rmk}

\begin{dfn}
\thmtitle{The Born-Oppenheimer approximation}
The \textit{Born-Oppenheimer approximation} decouples the electrons from the nuclei by neglecting the nuclear kinetic energy operator,
$
  \op{H}^\mr{bo}(\bo{r}_\mr{e},\bo{r}_\mr{n})
\equiv
  \op{H}(\bo{r}_\mr{e},\bo{r}_\mr{n})
-
  \op{T}^\mr{n}
$,
and treating nuclear attraction as a fixed external potential.
The resulting \textit{electronic Schr\"odinger equation} has the form
\begin{align}
\label{eq:born-oppenheimer-equation}
  \op{H}^\mr{bo}(\bo{r}_\mr{e},\bo{r}_\mr{n})
  \Y_k^\mr{e}(\bo{r}_\mr{e},\bo{r}_\mr{n})
=
  E_k^\mr{e}(\bo{r}_\mr{n})
  \Y_k^\mr{e}(\bo{r}_\mr{e},\bo{r}_\mr{n})
\end{align}
which is a continuous series of eigenvalue equations parametrized by $\bo{r}_\mr{n}$.
The index $k$ refers to the \textit{electronic state} of the molecule, which is a construct of the Born-Oppenheimer approximation.
The \textit{noncrossing rule} says that these electronic energies form non-intersecting \textit{potential energy surfaces} over the space of nuclear coordinates, $\{E_k^\mr{e}(\bo{r}_\mr{n})\,|\,\bo{r}_\mr{n}\}$.\footnote{
  More precisely, electronic states of a given spin and spatial symmetry do not intersect.
}
To construct a total molecular wavefunction, the Born-Oppenheimer approximation employs two simplifying assumptions.
\begin{align}
\label{eq:born-oppenheimer-approximations}
  \Y_\a^\mr{tot}
\approx
  \Y_{k\nu}^\mr{bo}
\ \ \text{where} \ \ 
  \Y_{k\nu}^\mr{bo}(\bo{r}_\mr{e},\bo{r}_\mr{n})
\equiv
  \Y_k^\mr{e}(\bo{r}_\mr{e},\bo{r}_\mr{n})
  \x_{k,\nu}^\mr{bo}(\bo{r}_\mr{n})
&&
  [\op{T}^\mr{n}, \Y_k^\mr{e}(\bo{r}_\mr{e},\bo{r}_\mr{n})]
\approx
  0
\end{align}
Applying these to the exact Schr\"odinger equation yields a \textit{rovibrational Schr\"odinger equation} for the nuclear component.
\begin{align}
\label{eq:born-oppeheimer-nuclear-motion-equation}
  (
    \op{T}^\mr{n}
  +
    E_k^\mr{e}(\bo{r}_\mr{n})
  )\,
  \x_{k,\nu}^\mr{bo}(\bo{r}_\mr{n})
=
  E_{k,\nu}^\mr{bo}\,
  \x_{k,\nu}^\mr{bo}(\bo{r}_\mr{n})
\end{align}
This scheme results in a good approximation to the wavefunction under the following two conditions:
1.~The nuclear wavefunction $\x_{k,\nu}(\bo{r}_\mr{n})$ is localized to a region where the potential energy surface $E_k(\bo{r}_\mr{n})$ is well-separated from its neighboring surfaces;
and 2.~The electronic wavefunction $\Y_k^\mr{e}(\bo{r}_\mr{e},\bo{r}_\mr{n})$ is slowly varying in $\bo{r}_\mr{n}$ over that region.
\end{dfn}

\begin{rmk}
\label{rmk:born-and-longuet-higgins-representations}
The electronic Schr\"odinger equation can also be used as a stepping stone to the exact molecular wavefunction, circumventing the approximations of equation~\ref{eq:born-oppenheimer-approximations}.
Namely, equation~\ref{eq:born-oppenheimer-equation} can be used to generate a convenient electronic basis for equations~\ref{eq:general-wavefunction-expansion}~and~\ref{eq:general-nuclear-schrodinger-equation}.
The dependence of the electronic states on $\bo{r}_\mr{n}$ admits two sensible choices for $\mc{H}^\mr{e}$.
\begin{align}
  {}^\mr{I}\mc{H}^\mr{e}
\equiv
  \{
    \Y_k^\mr{e}(\bo{r}_\mr{e},\bo{r}_\mr{n}^\circ)
  \,|\,
    k
  \}
&&
  {}^\mr{II}\mc{H}^\mr{e}
\equiv
  \{
    \Y_k^\mr{e}(\bo{r}_\mr{e},\bo{r}_\mr{n})
  \,|\,
    k
  \}
\end{align}
Option I is the \textit{Longuet-Higgins representation}, which uses Born-Oppenheimer electronic states from a fixed point on the potential surface.
Option II is the \textit{Born representation}, which associates each set of nuclear coordinates with its own set of electronic functions.
Both options approach the exact non-relativistic wavefunction in the limit of an infinite expansion, but the Born representation is more amenable to coarse-graining with a small number of functions.
The Longuet-Higgins representation is appropriate when the probability density of the nuclei is localized within a narrow region around $\bo{r}_\mr{n}^\circ$,
as might be observed at a deep well on the potential surface.
\end{rmk}

\begin{samepage}
\begin{rmk}
To contrast the two alternatives in \cref{rmk:born-and-longuet-higgins-representations}, express the Hamiltonian as
$
  \op{H}(\bo{r}_\mr{e},\bo{r}_\mr{n})
=
  \op{T}^\mr{n}
+
  \op{H}^\mr{bo}(\bo{r}_\mr{e},\bo{r}_\mr{n})
$
and consider its matrix elements for each set of electronic functions.
\begin{align}
  \op{\bo{H}}(\bo{r}_\mr{n})
=
  \op{\bo{T}}^\mr{n}
+
  \bo{H}^{\mr{bo}}(\bo{r}_\mr{n})
&&
  H_{k,k'}^\mr{bo}(\bo{r}_\mr{n})
\equiv
  \int
  d^{n_\mr{e}}\bo{r}_\mr{e}\,\,
  \F_k^*(\bo{r}_\mr{e})\,
  \op{H}^{\mr{bo}}(\bo{r}_\mr{e},\bo{r}_\mr{n})
  \F_{k'}(\bo{r}_\mr{e})
&&
  \op{T}_{k,k'}^\mr{n}(\bo{r}_\mr{n})
\equiv
  \int
  d^{n_\mr{e}}\bo{r}_\mr{e}\,\,
  \F_k^*(\bo{r}_\mr{e})\,
  \op{T}^\mr{n}\,
  \F_{k'}(\bo{r}_\mr{e})
\end{align}
Assuming the electronic basis states are orthonormal,\footnote{As a Hermitian operator, the Born-Oppenheimer model Hamiltonian always possesses an orthonormal set of eigenstates.} the matrix elements of $\op{T}^\mr{n}$ can be written as follows.\footnote{
Using
$
  \op{T}^\mr{n}\F_k(\bo{r}_\mr{n})
=
  \F_k(\bo{r}_\mr{n})\,\op{T}^\mr{n}
+
  [\op{T}^\mr{n}, \F_k(\bo{r}_\mr{n})]
$.
}
\begin{align}
  \op{T}_{k,k'}^\mr{n}
&=
  \d_{kk'}
  \op{T}^\mr{n}
+
  \op{\La}_{k,k'}^\mr{n}
&
  \op{\La}_{k,k'}^\mr{n}
&\equiv
  \int
  d^{n_\mr{e}}\bo{r}_\mr{e}\,\,
  \F_k^*(\bo{r}_\mr{e})\,
  [
    \op{T}^\mr{n}
  ,
    \F_{k'}(\bo{r}_\mr{e})
  ]
\\
  H_{k,k'}^\mr{bo}(\bo{r}_\mr{n})
&=
  \d_{kk'}
  E_k^\mr{e}(\bo{r}_\mr{n})
+
  \bar{H}_{k,k'}^\mr{bo}(\bo{r}_\mr{n})
&
  \bar{H}_{k,k'}^\mr{bo}(\bo{r}_\mr{n})
&\equiv
  \int
  d^{n_\mr{e}}\bo{r}_\mr{e}\,\,
  \F_k^*(\bo{r}_\mr{e})\,
  (\op{H}^{\mr{bo}}(\bo{r}_\mr{e},\bo{r}_\mr{n}) - E_k^\mr{e}(\bo{r}_\mr{n}))
  \F_{k'}(\bo{r}_\mr{e})
\end{align}
The structure of these matrices is characterized by the following relationships
\begin{align}
  [
    \op{T}^\mr{n}
  ,
    \Y_k^\mr{e}(\bo{r}_\mr{e},\bo{r}_\mr{n}^\circ)
  ]
=
  0
&&
  (\op{H}^{\mr{bo}}(\bo{r}_\mr{e},\bo{r}_\mr{n}) - E_k^\mr{e}(\bo{r}_\mr{n}))
  \Y_k(\bo{r}_\mr{e}, \bo{r}_\mr{n}^\circ)
\neq
  0
\\
  [
    \op{T}^\mr{n}
  ,
    \Y_k^\mr{e}(\bo{r}_\mr{e},\bo{r}_\mr{n})
  ]
\neq
  0
&&
  (\op{H}^{\mr{bo}}(\bo{r}_\mr{e},\bo{r}_\mr{n}) - E_k^\mr{e}(\bo{r}_\mr{n}))
  \Y_k(\bo{r}_\mr{e}, \bo{r}_\mr{n})
=
  0
\end{align}
which follow from equation~\ref{eq:born-oppenheimer-equation} and the fact that $\op{T}^\mr{n}$ is a differential operator in $\bo{r}_\mr{n}$.
In words, this says that the Longuet-Higgins representation diagonalizes the nuclear kinetic energy operator, whereas the Born representation diagonalizes the Born-Oppenheimer potential.
\end{rmk}
\end{samepage}

\begin{rmk}
In the Longuet-Higgins representation, the nuclear motion problem is typically solved by expanding the Hamiltonian in a Taylor series at $\bo{r}_\mr{n}^\circ$, which can be expressed as follows.
\begin{align}
  {}^\mr{I}\op{\bo{H}}(\bo{r}_\mr{n})
=
  \op{T}^\mr{n}\,
  \bo{1}
+
  \bo{E}^\mr{e}(\bo{r}_\mr{n})
+
  \tpd{
    \bo{H}^\mr{bo}(\bo{r}_\mr{n}^\circ)
  }{
    \bo{r}_\mr{n}
  }
  \cdot
  \d\bo{r}_\mr{n}
+
  \tfr{1}{2}\,
  \d\bo{r}_\mr{n}
  \cdot
  \tpd{
    ^2\bo{H}^\mr{bo}(\bo{r}_\mr{n}^\circ)
  }{
    \bo{r}_\mr{n}
    \pt
    \bo{r}_\mr{n}
  }
  \cdot
  \d\bo{r}_\mr{n}
+
  \ld
&&
  \bo{E}^\mr{e}(\bo{r}_\mr{n})
\equiv
  \ma{
    E_0^\mr{e}(\bo{r}_\mr{n}) & 0 & \cd \\
    0 & E_1^\mr{e}(\bo{r}_\mr{n}) & \cd \\
    \vd & \vd & \dd \\
  }
\end{align}
This approach leads to the K\"oppel-Domcke-Cederbaum (KDC) approximation to the total molecular wavefunction.
\end{rmk}

\begin{rmk}
The Born-representation Hamiltonian has the following structure.
\begin{align}
  {}^\mr{II}\op{\bo{H}}(\bo{r}_\mr{n})
=
  \op{T}^\mr{n}\,
  \bo{1}
+
  \op{\bm{\La}}^\mr{n}(\bo{r}_\mr{n})
+
  \bo{E}^\mr{e}(\bo{r}_\mr{n})
&&
  \op{\La}_{k,k'}^\mr{n}(\bo{r}_\mr{n})
=
  \int
  d^{n_\mr{e}}\bo{r}_\mr{e}\,\,
  \Y_k^{\mr{e}*}(\bo{r}_\mr{e},\bo{r}_\mr{n})\,
  [
    \op{T}^\mr{n}
  ,
    \Y_k^\mr{e}(\bo{r}_\mr{e},\bo{r}_\mr{n})
  ]
\end{align}
Although this approach leads to a more compact wavefunction expansion, it is considerably more complicated to work with because the elements of the coupling matrix $\op{\bm{\La}}^\mr{n}$ involve derivatives of the electronic states with respect to $\bo{r}_\mr{n}$.
\end{rmk}

\begin{dfn}
\thmtitle{The adiabatic approximation}
Neglecting the off-diagonal elements of the coupling matrix in the Born representation leads to the \textit{adiabatic approximation}, which is equivalent to employing the Born-Oppenheimer separability assumption,
$
  \Y_{k,\nu}^\mr{ad}(\bo{r}_\mr{e},\bo{r}_\mr{n})
=
  \Y_k^\mr{e}(\bo{r}_\mr{e},\bo{r}_\mr{n})
  \x_{k,\nu}^\mr{ad}(\bo{r}_\mr{n})
$,
without the vanishing commutator approximation.
The nuclear functions in the adiabatic approximation satisfy a set of conditions similar to equation~\ref{eq:born-oppeheimer-nuclear-motion-equation}.
\begin{align}
  (
    \op{T}^\mr{n}
  +
    \op{\La}_{k,k}^\mr{n}(\bo{r}_\mr{n})
  +
    E_k^\mr{e}(\bo{r}_\mr{n})
  )\,
  \x_{k,\nu}^\mr{ad}(\bo{r}_\mr{n})
=
  E_{k,\nu}^\mr{ad}\,
  \x_{k,\nu}^\mr{ad}(\bo{r}_\mr{n})
\end{align}
\end{dfn}

\begin{dfn}
\thmtitle{The diagonal Born-Oppenheimer approximation}
First-order perturbation theory leads to an even simpler alternative to the adiabatic approximation, known as the \textit{diagonal Born-Oppenhimer correction}.
\begin{align}
  E_{k,\nu}^\mr{ad}
\approx
  E_{k,\nu}^\mr{bo}
+
  \D_k^\mr{dboc}(\bo{r}_\mr{n})
&&
  \D_k^\mr{dboc}(\bo{r}_\mr{n})
=
  \int
  d^{n_\mr{e}}\bo{r}_\mr{e}\,\,
  \Y_k^{\mr{e}*}(\bo{r}_\mr{e},\bo{r}_\mr{n})\,
  \op{T}^\mr{n}\,
  \Y_k^\mr{e}(\bo{r}_\mr{e},\bo{r}_\mr{n})
\end{align}
\end{dfn}



\end{document}